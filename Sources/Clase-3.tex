\chapter{Clase 3: 04/09}

La mecánica estadística se justifica como consecuencia de un entrelazamiento
de un estado con su entorno.

Dado un estado entrelazado de 2 qubits

\[
\KET{\psi_{AB}}=\frac{\KET{00}+\KET{11}}{\sqrt{2}}\ACLARACION{\text{el de la derecha es un estado mezcla estadística}}{\neq}\frac{\PROYECT{00}{00}+\PROYECT{11}{11}}{2}
\]

Esta fundamentación general de mecánica estadística parte de que cuando
uno tiene este estado simple entrelazado y hacemos 
\[
Tr_{B}\PROYECT{\psi_{AB}}{\psi_{AB}}=Tr_{B}\PARENTESIS{\frac{\PARENTESIS{\KET{00}+\KET{11}}\PARENTESIS{\BRA{00}+\BRA{11}}}{2}}
\]

\[
=\sum_{\mu=0,1}\BRAKETEO{\mu_{0}}{\psi_{AB}}\BRAKETEO{\psi_{AB}}{\mu_{0}}=\frac{1}{2}\PARENTESIS{\PROYECT 00+\PROYECT 11}=\rho_{A}
\]

\[
\EXPECT{\psi_{AB}}{\psi_{AB}}{O_{A}\otimes I_{B}}=Tr\rho_{A}O_{A}
\]

Para ver los estados fundamentales se estudian sistemas de 2 qubits
haces trazas parcial y ya ves que se genera un estado mezcla (estado
térmico).

\section{Aplicaciones: }

\subsection{Teleportación cuántica (1993)}

Transferir un estado cuántico de un sistema local (Ej. remoto) a través
del entrelazamiento. Entonces es una nueva forma de comunicación. 

Parte de un circuito como el de la clase pasada haces evolucionar
el estado producto y sale un entrelazado como $\KET{\psi_{AB}}$ entonces
esos dos qubits que en su momento no interactuaron cuánticamente salen
entrelazados pero uno fue a un lugar y el otro a otro (los dos caminos
de abajo). Esto se puede usar para transmitir un estado? Propongo
un 3er estado $\KET{\varphi_{C}}=\alpha\KET 0+\beta\KET 1$.

El sistema parte de un estado (\textbf{etapa }I)
\[
\KET{\psi_{CAB}}=\PARENTESIS{\alpha\KET 0+\beta\KET 1}\PARENTESIS{\underset{AB}{\KET{00}}+\KET{11}}/\sqrt{2}
\]

Hacer alguna operación en el planeta A (involucra 2 canales cuánticos,
los dos de arriba) y que el estado $\KET{\psi_{C}}$ aparezca en el
planeta B. La idea es aplicar una medida de Bell (transforma estados
de Bell en estados productos). El entrelazamiento no es espacial si
están a largas distancias mientras el estado no interactuó con otra
cosa no se pierde.

\begin{quantikz}[slice all] \lstick[wires=2]{$\text{Planeta A}$} & \lstick{$\ket{\varphi_c}=\alpha\ket{0}+\beta\ket{1}$}& \ctrl{1} & \gate{H} & & \ctrl{2}&\qw\\ \qw&\qw&\octrl{-1}&\qw&\ctrl{1}&\qw&\qw\\ \lstick[wires=1]{$\text{Planeta B }\ket{\psi_{AB}}=\frac{\ket{00}+\ket{11}}{\sqrt{2}}$}&\qw&\qw&\qw&\gate{\sigma_x}&\gate{\sigma_x}&\rstick{$\ket{\varphi_c}$} 
\end{quantikz}
\begin{center}
\par\end{center}

\textbf{Etapa intermedia II }

Entre el Control-NOT y el Hadamard

\[
\frac{\alpha}{\sqrt{2}}\KET 0\PARENTESIS{\KET{00}+\KET{11}}+\frac{\beta}{\sqrt{2}}\KET 0\PARENTESIS{\KET{10}+\KET{01}}
\]

\textbf{Etapa III}

Después del Hadamard $H\KET 0=\frac{\KET 0+\KET 1}{\sqrt{2}}$ y $H\KET 1=\frac{\KET 0-\KET 1}{\sqrt{2}}$.
Entonces el estado evoluciona a 

\[
\frac{\alpha}{\sqrt{2}}\frac{\KET 0+\KET 1}{\sqrt{2}}\PARENTESIS{\KET{00}+\KET{11}}+\frac{\beta}{\sqrt{2}}\frac{\KET 0-\KET 1}{\sqrt{2}}\PARENTESIS{\KET{10}+\KET{01}}
\]

(todo esto hecho localmente en A)

Si juntamos todo 

\[
=\frac{1}{2}\CORCHETES{\underset{AC}{\KET{00}}\LLAVEABAJO{\PARENTESIS{\alpha\underset{B}{\KET 0}+\beta\underset{B}{\KET 1}}}{\KET{\varphi}}+\KET{01}\LLAVEABAJO{\PARENTESIS{\alpha\KET 1+\beta\KET 0}}{\sigma_{x}\KET{\varphi}}+\KET{10}\LLAVEABAJO{\PARENTESIS{\alpha\KET 0-\beta\KET 1}}{\sigma_{z}\KET{\varphi}}+\KET{11}\LLAVEABAJO{\PARENTESIS{\alpha\KET 1-\beta\KET 0}}{\sigma_{x}\sigma_{z}\KET{\varphi}}}
\]

Entonces si uno después de esto mide en A $\KET{00}$en B aparece
el $\KET 0$.

Lo que vemos es que como consecuencia el estado c o bien paso al otro
lado tal cual o bien paso al otro lado transformado con algún $\sigma_{\nu}$.
Entonces para hacer que el estado original aparezca en el otro planeta
hay 2 opciones:

- Si están cerca y sólo quiero transferir de un cable al otro es hacer
un control si esta $\KET{00}$ no hago nada o si hay $\KET{01}$ hago
$\sigma_{x}$, así sucesivamente (usando que $\sigma_{\nu}^{2}=I$->
anulo la transformación) y así tendría en B en el estado $\KET{\varphi}$.

Si pongo un Control-X $\sigma_{x}$ y un control z, Si está en 1 le
aplica $\sigma_{x}$ ahora en el otro si esta $\KET 1$ aplica $\sigma_{z}$
ahora bien si está en los dos entonces actual los dos. Entonces con
este esquema en la etapa cuatro obtengo en el último canal $\KET{\varphi_{c}}$.
Puedo decir que el entrelazamiento entre A y B se terminó porque se
usó para transferir el estado $\KET{\varphi_{B}}=\alpha\KET 0+\beta\KET 1$. 

Una forma efectiva es modificar el circuito haciendo una medida $\CheckedBox$,
luego esta $\sigma_{x}^{i}$ siendo $i=0,1$ entonces aplica o no
aplica, esto significa una comunicación clásica (la raya curvadita),
llamo y le digo tenés que aplicar la medida. Hago lo mismo con $\sigma_{z}$.
\begin{center}
\includegraphics[scale=0.7]{\string"fig2 clase4-9\string".png}
\par\end{center}

Se usa cristal birrefringente uno de los fotones queda en la tierra
el otro se manda al satélite se hacen las cosas y se prueba esto.

Esto muestra que el entrelazamiento nos permite una nueva forma de
comunicación.

Si hago traza parcial en el canal B antes de los sigmas, 
\[
\rho_{B}=Tr_{AC}\PROYECT{\psi_{CAB}}{\psi_{CAB}}=\frac{\PROYECT 00+\PROYECT 11}{0}
\]

Luego voy a obtener el estado puro $\KET{\varphi}$.

Si hubiera un estado D entrelazado con C, luego de aplicar todo esto
terminaría teniendo el estado D entrelazado con B.

\subsection{Codificación Superdensa}

¿Qué información puede transmitirse con un qubit? La idea es enviar
un qubit. 

Dada una fuente que emite un par de fotones o qubits entrelazados
\[
\KET{\psi_{AB}}=\frac{\KET{00}+\KET{11}}{\sqrt{2}}
\]

En A hacemos una operación U sobre ese qubit. Luego vuelve y hacemos
una medida conjunta en estos dos cables cuánticos.
\begin{center}
\includegraphics[scale=0.5]{\string"fig3 clase4-9\string".png}
\par\end{center}

Si por ej A aplica la I entonces el estado queda igual. 

Si A aplica $\sigma_{x}$, entonces $\KET{\psi_{AB}}\rightarrow\frac{\KET{10}+\KET{01}}{\sqrt{2}}$

Si A aplica $\sigma_{z}$, entonces $\KET{\psi_{AB}}\rightarrow\frac{\KET{00}-\KET{11}}{\sqrt{2}}$

Si A aplica $\sigma_{y}=\sigma_{x}\sigma_{z}$, $\KET{\psi_{AB}}\rightarrow\frac{\KET{10}-\KET{01}}{\sqrt{2}}$

Luego hago la medida conjunta y veo que llego a 4 estados ortogonales
bien distinguidos, entonces si puedo medir estados Bell, pero si recibo
el qubit que viene de A como esta entrelazado voy a poder ver la operación
que hice y voy a obtener 2 bits de información 1 de la operación que
hice y 1 del estado.

Entonces vemos que el entrelazamiento lo podemos usar para transmitir
información. 

Para medir en base estándar mido la polarización directo o con un
stern y gerlach. Ahora bien si quiero hacer medida en la base de Bell,
lo que hago antes de medirlo en la base computacional le aplico alguna
operación para pasarlo de la base de Bell a la computacional: primero
control not y después la Hadamard 
\begin{center}
\includegraphics[scale=0.5]{\string"fig4 clase4-9\string".png}
\par\end{center}

Nosotros sabemos de álgebra que $(AB)^{-1}=B^{-1}A^{-1}$pero dado
que la inversa del Control-NOT es sí mismo y de la Hadamard
es sí misma también, por eso podemos invertir como indica la figura. 

%%%%%%%%%%%%%%%%%%%%Hasta acá esta en algoritmos%%%%%%%%%%%%%%

\section{Demostración de que el desorden global es mayor o igual que el desorden
local en todo sistema descripto por variables aleatorias. (esto es
clásico)}

Tenemos un sistema (A,B) conjunto, entonces tiene una distribución
de probabilidad conjunta
\[
P_{ij}=P(A=i,B=j)
\]
 
\[
S(A,B)=-\sum_{i,j}P_{ij}\log P_{ij}\qquad\text{Medida del desorden global}
\]

\[
S(A,B)=0\iff P_{ij}=\delta_{i}\delta_{j}
\]

\[
S(A,B)\:max\iff P_{ij}=\frac{1}{\sqrt{n_{a}n_{b}}}
\]

Distribución marginal que es $P_{i}^{A}=\sum_{j}P_{ij}$

Entropía de A: $S(A)=-\sum_{i}P_{i}^{A}\log P_{i}^{A}$

Entropía condicional clásica de B dado A $S(B|A)=\sum_{i}P(A=i)\LLAVEABAJO{\sum_{j}P(B=j|A=i)\log P(B=j|A=i)}{S(B|A=i)}$

$P(B=j|A=i)=\frac{P(B=j,A=i)}{P(A=i)}=\frac{P_{ij}}{P_{i}^{A}}$ la
coma significa intersección. 

\[
S(B|A)=-\sum_{i}P_{i}^{A}\sum_{j}\frac{P_{ij}}{P_{i}^{A}}\log\frac{P_{ij}}{P_{i}^{A}}=-\sum_{ij}P_{ij}\PARENTESIS{\log\PARENTESIS{P_{ij}}-\log\PARENTESIS{P_{i}^{A}}}
\]

\[
=S(A,B)-\PARENTESIS{-\sum_{i}P_{i}^{A}\log P_{i}^{A}}=S(A,B)-S(A)\geq0
\]

Que sea $\geq0$ implica que $S(A,B)\geq S(A)$. Entonces el desorden
global es mayor o igual que el desorden local. $\blacksquare$

Ahora bien en un estado de Bell 
\[
\KET{\psi_{AB}}=\frac{\KET{00}+\KET{11}}{\sqrt{2}}
\]
 
\[
\rho_{AB}=\PROYECT{\psi_{AB}}{\psi_{AB}}\qquad\rho_{A}=\frac{\PROYECT 00+\PROYECT 11}{\sqrt{2}}
\]
Ahora ya no tenemos sumas parciales tenemos trazas parciales.
\[
S(A,B)=S(\text{\ensuremath{\rho_{AB}}})=0
\]

Pero ahora $S(\rho_{A})=1=S(A)$ entonces cuánticamente no se cumple
esto.

La entropía condicional cuántica ya no es positiva. 

Para el caso de operadores densidad diagonales:

\[
\rho_{AB}=\sum_{ij}\PROYECT{ij}{ij}p_{ij}
\]
en una base producto (con $p_{ij}>0$), la entropía conjunta se comporta
como una clásica. Tengo un <<mundo clásico>>.

\section{Computación Cuántica}

Idea fundamental (Feymann):

\textbf{Paralelismo cuántico} en un paso puede hacer 2 operaciones
o si más 
\begin{center}
\includegraphics[scale=0.7]{\string"fig5 clase4-9\string".png}
\par\end{center}

Superponer los dos estados es natural es cambiar la dirección del
estado. 

\subsection{El primer algoritmo (histórico) Deutsch 1986.}

El problema más simple posible donde la mecánica cuántica pueda hacer
algo que lo clásico no, entonces dada una función $f:\LLAVES{0,1}\rightarrow\LLAVES{0,1}$
y la pregunta que me hago: $f(0)=f(1)$ o $f(0)\neq f(1)$ que clásicamente
necesita dos evaluaciones de la función pero cuánticamente lo podemos
hacer en una sola evaluación.

Planteo un sistema de dos qubits pero pongo un bloque grande que implica
una operación no sobre un sólo qubit sino una operación que involucra
ambos qubits. Es decir hacerlo con una sóla entrada

Sabemos que $f(0)=0\:o\:1$ al igual que $f(1)$. $U_{f}\KET{ij}=\KET{i,j\oplus f(i)}$
con $\oplus$ suma en módulo 2. Podemos ver que $U_{f}$ es unitaria. 
\begin{center}
\includegraphics[scale=0.75]{\string"fig6 clase4-9\string".png}
\par\end{center}

Entonces qué pasa 
\[
\KET 0\PARENTESIS{\frac{\KET 0-\KET 1}{\sqrt{2}}}=\frac{\KET{00}-\KET{01}}{\sqrt{2}}\underset{U_{f}}{\rightarrow}\frac{U_{f}\KET{00}-U_{f}\KET{01}}{\sqrt{2}}
\]

\[
=\KET{0,f(0)}-\KET{0,1\oplus f(0)}=\KET 0\PARENTESIS{\KET{f(0)}-\KET{1\oplus f(0)}}=(-1)^{f(0)}\KET 0\frac{\KET 0-\KET 1}{\sqrt{2}}
\]

Si $f(0)$ es 0 queda el estado original

Si $f(0)=1$ queda - el estado original

Ahora bien el resultado que uno obtiene de esta suma

\[
\frac{1}{2}\PARENTESIS{\KET 0(-1)^{f(0)}+\KET 1(-1)^{f(1)}}\PARENTESIS{\KET 0-\KET 1}
\]
entonces uno obtiene el valor de $f$ como una fase. 

\[
\frac{1}{2}(-1)^{f(0)}\PARENTESIS{\KET 0+\KET 1(-1)^{f(1)-f(0)}}\PARENTESIS{\KET 0-\KET 1}
\]
Ahí vimos que se puede hacer en un sólo paso. La fase que tenemos
sobre todo el estado no nos importa, ahora bien si $f(0)=f(1)$ entonces
me devuelve el estado original, ahora si $f(0)\neq f(1)$ me da el
estado ortogonal. Mido en la base de x entonces si me da 1 son iguales
si me da -1 son distintas. Vuelvo a ponerlo en la base computacional
con una Hadamard, y obtengo
\begin{center}
\includegraphics[scale=0.7]{\string"fig7 clase4-9\string".png}
\par\end{center}
