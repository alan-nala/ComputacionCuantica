%%%%%%%%%%%% Lista de Temas de la Notas que ya se agregaron al texto

# Teleportación Cuántica
# Codificación Superdensa


% 1. Postulados de la mecánica cuántica
2. Entrelazamiento:
2.0.1. ¿Cómo diagonalizar una matriz
2.1. Descomposición de Schmidt, . . . . .
2.2. Observable local en A: . . . . . . . .
2.3. Operador densidad . . . . . . . . . . .
2.4. Matriz densidad reducida: . . . . . .
2.5. Entropía de entrelazamiento (E) . . .
II

3
cuadrada?
.
Clase 2 - 02 - 09

3. Entropía
3.1. Entropía de Von Neumann . . . . . . . . . . . . . . . . . . . . . . . . . . . . . . 
3.2. Entropía de Shannon . . . . . . . . . . . . . . . . . . . . . . . . . . . . . . . . 
3.3. Propiedades de la entropía . . . . . . . . . . . . . . . . . . . . . . . . . . . . .
4. Qubits
4.1. Estado general de un qubit . . . . . . . . . . . . . . .
4.2. No cloning theorem . . . . . . . . . . . . . . . . . . 
4.3. Representación gráfica de un estado de un qubit . . .
4.4. Bases: . . . . . . . . . . . . . . . . . . . . . . . . 
4.4.1. Base computacional: . . . . . . . . . . . . . . .
4.4.2. Base de Bell: formada por estados entrelazados:
12
% 5. Compuertas lógicas:
% 5.1. Hadamard . . . . . . . . . . . . . . . . . . . . . . . . . . . . . . . . . . . . . . . . . 12
% 13
III 13
.
Clase 3: 04/09
6. Aplicaciones:
% 6.1. Teleportación cuántica (1993) . . . . . . . . . . . . . . . . . . . . . . . . . . . . . .
% 6.2. Codificación superdensa . . . . . . . . . 
% 7. Demostración de que el desorden global es mayor o igual que el desorden local en todo sistema descripto por variables aleatorias. (esto es clásico)
% 16
% 18. Computación cuántica
% 8.1. El primer algoritmo (histórico) Deutsch 1986. . . . . . . . . . . . . . . . . . . . . . 
% IV 19
% Clase 4: 09/09
% 9. Algoritmo de Deutsch 19
% 10.Algoritmo de Deutsch - Joscz 19
11.Medidas generalizadas
11.0.1. Ejemplo qubit . . . . . . . . . . . . . . . . . . . . . . . . . . . . . . . . . .
11.0.2. Ej. distinción de estados no ortogonales . . . . . . . . . . . . . . . . . . . .
11.1. ¿Qué es una medida? . . . . . . . . . . . 
V Clase 11/09 23
12.Medidas locales 23
13.Paradoja de EPR (1935) 25
14.Desigualdad de Bell
14.1. CHSH: versión simplificada . . . . . . . 27
VI 29
Clase 6: 16/09
15.Compuertas
15.1. Hadamard . . . . . . . . . . . . . . . . . . . . . . . . . . . . . . . . . . . . . . . . .
15.2. Control not: circuito de 2 qubits . . . . . . . . . . . . . . . . . . . . . . . . . . . . 16.Estados no puros
16.1. Ejemplo de tomografía de estados cuánticos. . . . . .
16.2. Cómo los generamos . . . . . . . . . . . . . . . . . .
16.3. Estados no puros de 2 qubits . . . . . . . . . . . . .
16.4. Entrelazamiento de estados no puros . . . . . . . . .
16.4.1. Estados separables no puro . . . . . . . . . .
16.5. ¿Cómo sabemos si este estado no puro es entrelazado
% 16.5.1. Criterio de Peres (PPT) . . . . . . . . . . . .
% 16.6. Medida de negatividad . . . . . . . . . . . . . . . . .
% 16.7. Ejemplo típico del lab . . . . . . . . . .
% 16.8. Criterios necesarios de separabilidad . . . . . . . . . . . . . . . . . .
% 16.8.1. Traspuesta parcial . . . . . . . . . . . . . . . . . . . . . . . .
% 16.8.2. Negatividad . . . . . . . . . . . . . . . . . . . . . . . . . . . .
% 16.8.3. Criterio de desorden (o entrópico) de separabilidad . . . . . .
% 16.9. Testigo de entrelazamiento . . . . . . . . . . . . . . . . . . . . . . . .
% 16.9.1. Demostración . . . . . . . . . . . . . . . . . . . . . . . . . . .
% 16.9.2. Ejemplo . . . . . . . . . . . . . . . . . . . . . . . . . . . . . .
% 16.10.¿Cómo mido el valor medio de entrelazamiento de estados no puros?
% 16.10.1.Entrelazamiento de formación . . . . . . . . . . . . . . . . . .
% 16.10.2.Fórmula de Wooters (2 qubits) . . . . . .
% VII
% . . . . . . . .
% o separable?
% . . . . . . . .
Clase 7: 18/09
3716.11.Algoritmos para estados mezcla - Knill LaFlamme - intro . . . . . . . . . . . . . . .
VIII
Clase 8: 25/09
38
38
17.Algoritmos para estados mezcla - Milly Laflamme 1998 (Deterministic quantum
computation with one qubit - DQCL)
38
17.1. ¿Qué pasa si el qubit de control en vez de ser puro es mezcla? . . . . . . . . . . . . 39
18.Información mutua
18.1. Clásicamente . . . . . . . . . . . . . . . . . . . . . . . . . . . . . . . . . . . . . . .
18.2. Cuántico . . . . . . . . . . . . . . . . . . . . . . . . . . . . . . . . . . . . . . . . . .
18.2.1. Ej de información mutua: Estado de Bell 19.Discord (Zureh 2001)
19.1. ¿Cuándo se anula? . . . . . . .
19.2. ¿Qué pasa para estados puros?
19.3. Discord e información mutua .
19.4. Ej - ejercicio de la práctica . .
19.5. Propiedades del discord . . . .
19.6. Monogamia del entrelazamiento
IX Clase 10: 30/09 44

19.Discord (Zurek 2001) 44
19.1. ¿Cuándo se anula? . .
19.2. ¿Qué pasa para estados puros? . .
19.3. Discord e información mutua . . 
19.4. Ej - ejercicio de la práctica . .
19.5. Propiedades del discord . . .
19.6. Monogamia del entrelazamiento - problema 3 de la práctica 3 
19.6.1. Para un par de qubits. . . . . 48

X Clase 10: 02/10 48
20.Compuertas de un qubit 49
20.1. Ej el CNOT . . . . .
20.2. Operador de Swap . . . . . .
20.3. Universalidad del control not . . . . .
20.3.1. ¿Cómo hacemos un control U? . . . . . 
20.3.2. ¿Cómo logramos un control doble? . . . . 
20.3.3. ¿Cómo hacemos un control triple? . . . . . 
20.3.4. ¿Cómo hacemos una unitaria general de 2 qubits? . 
20.3.5. ¿Cómo hacemos una unitaria general? . . . . . . . .

XI Clase 11: 07/10 56
21.Algoritmo de Grover (Lov Kumar Grover - PRL QM helps in searching for a needle in a haystack.) 56
22.Algoritmo de Shor para Transformada de Fourier 60
22.1. Transformada de Fourier Discreta Clásica . . . .
22.2. Transformada de Fourier Discreta Cuántica . . . . . 
XII Clase 12: 09/10 61
3
23.Transformada de Fourier Cuántica (QFT) 61
23.1. Transformada Hadamard . . . . . . . . . .
23.2. Volvamos a la QFT . . . . . . . . . . . .
23.2.1. Operador de traslación . . . . . . . . .
23.2.2. Punto importante . . . . . . . . . . . 

XIII Clase 13: 14/10 64
23.3. Aplicación: Determinación del período . . . . . . 
23.4. Factorización . . . . . .
24.Evolución de sistemas abiertos: decoherencia 71
24.1. Representación de Kraus de la evolución de sistemas abiertos.
24.1.1. Dem: . . . . . . . 