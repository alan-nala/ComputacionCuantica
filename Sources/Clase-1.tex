\chapter{Clase 1 - 28 - 08}

\section{Postulados de la mecánica cuántica}

\textbf{1)} \textbf{Sistema aislado}: para cada sistema cuántico uno
le asocia un espacio de Hilbert $\mathcal{H}$de dimensión finita normalizado
con un ket $\KET{\psi}\in\mathcal{H}$

\textbf{2) Medidas:} es un conjunto de operadores $\LLAVES{M_{k},\sum_{k}M_{k}^{\dagger}M_{k}=I}$

Se satisface

$P(k)=\EXPECT{\psi}{\psi}{M_{k}^{\dagger}M_{k}}$, estas dos ecuaciones implican
que $\sum_{k}P(k)=1$

Luego de la medida $\KET{\psi}\rightarrow\frac{M_{k}\KET{\psi}}{\sqrt{P(k)}}$

Medidas proyectivas, caso particular de estas medidas, corresponde
al caso de que los $M_{k}$ son proyectores ortogonales: $M_{k}=P_{k}$,
$P_{k}^{2}=P_{k}$ , $P_{k}P_{l}=\delta_{kl}P_{k}$

\textbf{3)} Para sistema aislado el estado va a \textbf{evolucionar}
de forma unitaria: $\KET{\psi(t_{2})}=U_{t_{2}t_{1}}\KET{\psi(t_{1})}$
en información cuántica no pensamos la evolución temporal como algo continuo
sino más bien como un operador unitario que evoluciona. Si tomamos
como algo continuo llegamos a la ecuación de Schrödinger del operador
unitario.

\textbf{4) Sistemas compuestos }por varios subsistemas: 

\[
\mathcal{H}=\mathcal{H}_{1}\otimes\mathcal{H}_{2}\dots\otimes\mathcal{H}_{n}
\]

\[
\KET{\psi}\in\mathcal{H}
\]

\[
n=2\rightarrow\mathcal{H}=\mathcal{H}_{1}\otimes\mathcal{H}_{2}
\]

Las bases

\[
B_{1}=\LLAVES{\KET{1_{a}}}\quad\text{base de }\mathcal{H}_{1}
\]

\[
B_{2}=\LLAVES{\KET{1_{b}}}\text{base de }\mathcal{H}_{2}
\]

Base de $\mathcal{H}_{1}\otimes\mathcal{H}_{2}=\LLAVES{\KET{1_{a}}\otimes\KET{1_{b}}=\KET{1_{j}}}$

Entonces la $dim\mathscr{\mathcal{H}}=dim\mathscr{\mathcal{H}}_{1}.dim\mathscr{\mathcal{H}}_{2}$

\[
dim\mathscr{\mathcal{H}}=\PARENTESIS{dim\mathscr{\mathcal{H}}_{1}}^{n}
\]

Nos detenemos en este caso, cómo son los estados de este producto
tensorial, queremos puntualizar en el concepto de entrelazamiento.


\section{Entrelazamiento:}

\[
\mathcal{H}=\mathscr{\mathcal{H}}_{1_{A}}\otimes\mathscr{\mathcal{H}}_{2_{B}}
\]

sistemas compuestos: vamos a tomar subsistemas distinguibles. (Los
casos indisttinguibles los podemos hacer tender a sistemas distinguibles
y lo vamos a ver más adelante)

Tenemos estados productos:

\[
\KET{\psi_{AB}}=\KET{\psi_{A}}\otimes\KET{\psi_{B}}\ACLARACION{\text{aligeramos notación}}=\KET{\psi_{A}}\KET{\psi_{B}}
\]

En general: 
\[
\KET{\psi_{AB}}=\sum_{ij}C_{ij}\LLAVEABAJO{\KET{i_{A}}\otimes\KET{j_{B}}}{\KET{ij}}
\]

Si $C_{ij}=\alpha_{i}\beta_{j}$$\KET{\psi_{AB}}=\sum_{i}\PARENTESIS{\alpha_{i}\KET{i_{A}}}\PARENTESIS{\beta_{j}\KET{j_{B}}}=\KET{\psi_{A}}\KET{\psi_{B}}$

$\KET{\psi_{A}}=\sum_{i}\alpha_{i}\KET{i_{A}}$, $\KET{\psi_{B}}=\sum_{j}\beta_{j}\KET{j_{B}}$

\textbf{Estados entrelazados:} (no son estados producto)

\[
\KET{\psi_{AB}}\neq\KET{\psi_{A}}\KET{\psi_{B}}
\]
\emph{Uno no puede hacer combinaciones lineales de los estados que
tiene pero si de los estados producto que tiene.}

Un estado no producto: 
\[
\KET{\psi_{AB}}=\frac{\KET{0_{A}}\KET{0_{B}}+\KET{1_{A}}\KET{1_{B}}}{\sqrt{2}}
\]

esto es una suma de productos pero no es estado producto, esto no
es clásico, ej spines.

Caso más simple, spin 1/2. (Producto)

\[
\KET{\uparrow\uparrow}=\KET{\uparrow}\KET{\uparrow}
\]

ahora bien el estado entrelazado:

\[
\frac{\KET{\uparrow\uparrow}+\KET{\downarrow\downarrow}}{\sqrt{2}}
\]

\[
\BRAKETEO{i_{A}j_{B}}{k_{A}l_{B}}=\BRAKETEO{i_{A}}{k_{A}}\BRAKETEO{j_{B}}{l_{B}}
\]

\[
\KET{\psi_{AB}}=\sum_{ij}C_{ij}\KET{ij}
\]

\[
d_{A/B}=dim(\text{\ensuremath{\mathcal{H}}}_{A/B})
\]

Estado producto: corresponden al caso particular en el que esta matriz
tiene rango 1 $rango(C)=1$ $C:d_{A}\times d_{B}$

\[
C_{ij}=\alpha_{i}\beta_{j}=\left(\begin{array}{c}
\alpha_{i}\\
\vdots\\
\alpha_{d_{A}}
\end{array}\right)\PARENTESIS{\begin{array}{ccc}
\beta_{j} & \dots & \beta_{d_{B}}\end{array}}
\]


\subsubsection{¿Cómo diagonalizar una matriz cuadrada?}

Uno no puede diagonalizar así no más la matriz porque no es cuadrada:

Cualquier matriz se puede escribir como: 
\begin{equation}
\LLAVEABAJO C{d_{A}\times d_{B}}=\LLAVEABAJO U{d_{A}\times d_{A}}\LLAVEABAJO D{d_{A}\times d_{B}}\LLAVEABAJO{V^{\dagger}}{d_{B}\times d_{B}}\label{eq:Cij}
\end{equation}

$D$ diagonal, $U$ y $V$ son unitarias. Tener D diagonal podemos
usarla para escribir mejor $C_{ij}$

\[
D_{kk'}=\delta_{kk'}\sigma_{k}
\]

\[
\sigma_{k}\geq0
\]

\[
C_{ij}=\sum_{k}U_{ik}\sigma_{k}V_{jk}^{*}
\]
a partir de esto todo estado:

\[
\KET{\psi_{AB}}=\sum_{ij}C_{ij}\KET{i_{A}}\KET{i_{B}}
\]

\[
\BRAKETEO{\psi_{AB}}{\psi_{AB}}=1=\sum_{ij}\MODULO{C_{ij}}^{2}=1
\]

Suponemos bases ortogonales

\[
\KET{\psi_{AB}}=\sum_{ijk}\sigma_{k}U_{ik}\KET{i_{A}}V_{jk}^{*}\KET{j_{B}}
\]

\[
=\sum_{k}\sigma_{k}\LLAVEABAJO{\PARENTESIS{\sum_{i}U_{ik}\KET{i_{A}}}}{\KET{k_{A}}}\LLAVEABAJO{\PARENTESIS{\sum_{j}V_{jk}^{*}\KET{j_{B}}}}{\KET{k_{B}}}
\]

\[
=\sum_{k=1}^{n_{s}}\sigma_{k}\KET{k_{A}}\KET{k_{B}}
\]


\subsection{Descomposición de Schmidt}

entonces una descomposición de estados productos uno los puede escribir
con el mismo índice. $n_{s}$ es el número de Schmidt es el rango
de la matriz $C$ (que es igual al de $D$). Los estados productos
son el caso especial donde $n_{s}=1$ $\rightarrow$ \textbf{estado
puro clásico}

Usando la ortogonalidad de los i y de los j llegamos a la ortogonalidad
de los k $\BRAKETEO{k_{A}}{k'_{A}}=\delta_{kk'}$

Pasemos en limpio:

\[
\KET{\psi_{AB}}=\sum_{k=1}^{n_{s}}\sigma_{k}\KET{k_{A}}\KET{k_{B}}
\]

Con $\sum_{k}\sigma_{k}^{2}=1$ acá tenemos 2 casos

$\begin{cases}
\begin{array}{c}
n_{s}=1\rightarrow\KET{\psi_{AB}}=\KET{1_{A}}\KET{1_{B}}\\
n_{s}\geq2
\end{array} & \begin{array}{c}
\text{Estados producto}\\
\text{Estados entrelazados}
\end{array}\end{cases}\begin{array}{c}
\text{sistemas clásicos}\\
\text{sistemas puramente cuánticos}
\end{array}$

\[
n_{s}=rango(C)\leq Min(d_{A},d_{B})
\]

B es el resto del universo, que es muy complejo, pero si el sistema
A vive en un espacio de dimensión 2 entonces el estado más general
posible vive en dimensión 2. Esto indica que si las componentes de
un sistema es de dimensión reducida uno tiene una forma compacta de
escribir el sistema. 

\textbf{¿Cómo calculamos los $\sigma_{k}$?}

De la expresión \ref{eq:Cij} $\underset{d_{B}\times d_{B}}{\underset{d_{B}\times d_{A}}{C^{T}}\underset{d_{A}\times d_{B}}{C}}=VD^{T}U^{T}UDV^{\dagger}=VD^{T}DV^{\dagger}$

\[
\sigma_{k}^{2}=\lambda_{k}\PARENTESIS{C^{T}C}
\]

Análogo con $\underset{d_{A}\times d_{A}}{CC^{T}}=UDD^{T}U^{T}=\lambda_{k}\PARENTESIS{CC^{T}}$ 

\[
\sigma_{k}^{2}=\lambda_{k}\PARENTESIS{CC^{T}}
\]

vamos a obtener solo 2 autovalores, entonces para encontrar los sigmas
los podemos obtener a partir de diagonalizar una matriz de 2x2.

Supongamos el sistema $E$

\[
\mathcal{H}_{A}=\LLAVES{\KET{\uparrow_{A}},\KET{\downarrow_{A}}}
\]

\[
\mathcal{H}_{B}=\LLAVES{\KET{\uparrow_{B}},\KET{\downarrow_{B}}}
\]

Estado singlete, vemos que no es estado producto porque es descomposición
de Schmit

\[
\frac{\KET{\uparrow\downarrow}-\KET{\downarrow\uparrow}}{\sqrt{2}}=\KET{\psi_{AB}}
\]

estado de $S=0$

Estado triplete $S=1$

\[
\frac{\KET{\uparrow\downarrow}+\KET{\downarrow\uparrow}}{\sqrt{2}}
\]

$\frac{\KET{\uparrow\uparrow}\pm\KET{\downarrow\downarrow}}{\sqrt{2}}$
estos tienen $S=\pm1$

Estos son todos ej de estados entrelazados

La $dim\mathcal{H}_{A}=dim\mathcal{H}_{B}=2$ entonces $dim\mathcal{H}_{A}\otimes\mathcal{H}_{B}=4$

Estos estados son una base del sistema.

\subsection{Observable local en A: }

es un operador que corresponden a un observable que son de la forma:
$O_{A}\otimes I_{B}$ (Análogo para B)

$\EXPECT{\psi_{AB}}{\psi_{AB}}{O_{A}\otimes I_{B}}=\sum_{k,k'}\sigma_{k}\sigma_{k'}\LLAVEABAJO{\EXPECT{k_{A}k_{B}}{k_{A}'k'_{B}}{O_{A}\otimes I_{B}}}{\EXPECT{k_{A}}{k'_{A}}{O_{A}}\LLAVEABAJO{\BRAKETEO{k_{B}}{k'_{B}}}{\delta_{kk'}}}$

\[
=\sum_{k}\sigma_{k}^{2}\EXPECT{k_{A}}{k_{A}}{O_{A}}
\]
El estado está en el estado $k_{A}$ con probabilidad $\sigma_{k}^{2}$.

\[
\EXPECT{\psi_{AB}}{\psi_{AB}}{O_{A}\otimes I_{B}}=\text{Tr}\rho_{A}O_{A}
\]
 con $\rho_{A}=\sum_{k}\sigma_{k}^{2}\KET{k_{A}}\BRA{k_{A}}$ este
es el estado reducido del sistema, es un operador densidad.

Uno parte de un estado puro, es un estado de un sistema con T=0, el
estado fundamental del hamiltoniano conjunto. El valor medio del estado
puro viene de la mezcla estadística de estados puros, y esto corresponde
a estados calientes. Es un promedio local de un sistema puro entrelazado.
Un estado total que esta a temperatura 0, cuando uno va al sistema
local lo ve a temperatura distinta de cero, lo puedo ver como un estado
térmico. Uno no puede representar con función de onda el sistema local,
lo tenemos que hacer con un operador densidad (local), estos $\sigma_{k}^{2}$
son los autovalores de los autovectores $\KET{k_{A}}\BRA{k_{A}}$
que surgen de diagonalizar $\rho_{A}$. De forma análoga obtenemos
para $B$:

\[
\EXPECT{\psi_{AB}}{\psi_{AB}}{I_{A}\otimes O_{B}}=\text{Tr}\rho_{B}O_{B}
\]

\[
\rho_{b}=\sum_{k}\sigma_{k}^{2}\KET{k_{B}}\BRA{k_{B}}
\]

Otra de las cosas notables es que el subsistema total esta T=0, los
subsistemas estan calientes, con los mismo autovalores. Los subsistemas
tienen la misma entropía, porque la obtenemos a partir de los $\sigma_{k}$,
$S_{A}=S_{B}$.

\subsection{Operador densidad}

1Cuando uno calcula $\EXPECT{\psi}{\psi}O=Tr\PARENTESIS{\KET{\psi}\BRA{\psi}O}=\sum_{j}\BRAKETEO k{\psi}\EXPECT{\psi}jO$ 

\textbf{Estados puros} corresponden a un \textbf{operador densidad
$\rho=\KET{\psi}\BRA{\psi}$}, proyector ortogonal, satisfaciendo:
$\rho^{2}=\rho$ de rango uno.

En general, un operador densidad general no es un estado puro, es decir
$\rho^{2}\neq\rho$

$\rho=\sum_{i}p_{i}\KET{\psi_{i}}\BRA{\psi_{i}}$, $\BRAKETEO{\psi_{i}}{\psi_{j}}=\delta_{ij}$$p_{i}\geq0$
(esto último $\iff$) 

$\EXPECT{\psi}{\psi}{\rho}=\sum_{i}p_{i}\MODULO{\BRAKETEO{\psi}{\psi}}^{2}\geq0$

Para describir cosas locales necesito operadores densidad porque los
estados puros <<quedan chicos>>. 

\subsection{Matriz densidad reducida: }

dado un estado puro

\[
\KET{\psi_{AB}}\rightarrow\rho_{AB}=\KET{\psi_{AB}}\BRA{\psi_{AB}}
\]
tal que $\EXPECT{\psi_{AB}}{\psi_{AB}}{O_{AB}}=Tr\PARENTESIS{\rho_{AB}O_{AB}}$
con $\rho_{AB}^{2}=\rho_{AB}$

Vamos a lo local:

\[
\EXPECT{\psi_{AB}}{\psi_{AB}}{O_{A}\otimes I_{A}}=Tr\rho_{A}O_{A}
\]
 $\rho_{A}^{2}<\rho_{A}$el menor entre operadores implica que los
autovalores son menores

\[
\KET{\psi_{AB}}=\sum_{k}\sigma_{k}\KET{k_{A}}\KET{k_{B}}
\]

\[
\rho_{A}=\sum_{k}\sigma_{k}^{2}\KET{k_{A}}\BRA{k_{A}}\ACLARACION{\text{traza parcial}}=Tr_{B}\KET{\psi_{AB}}\BRA{\psi_{AB}}
\]

\[
=\sum\LLAVEABAJO{\BRAKETEO{k_{B}}{\psi_{AB}}}{\sigma_{k}\KET{k_{A}}}\LLAVEABAJO{\BRAKETEO{\psi_{AB}}{k_{A}}}{\sigma_{k}\KET{k_{B}}}
\]

\[
\rho_{A}=Tr_{B}\rho_{AB},\qquad\rho_{B}=Tr_{A}\rho_{AB}
\]

\[
Tr_{AB}\rho_{AB}(O_{A}\otimes I_{B})=Tr_{A}(\rho_{A}O_{A})
\]

\[
Tr_{AB}\rho_{AB}(I_{A}\otimes O_{B})=Tr_{B}(\rho_{B}O_{B})
\]

Esto lo tuvimos que introducir (estas trazas parciales ) debido a
que tenemos estados entrelazados.

$\rho_{AB}=\KET{\psi_{AB}}\BRA{\psi_{AB}}$ Puro $\rho_{AB}^{2}=\rho_{AB}$

$\rho_{A}=Tr_{B}\rho_{AB}$ no puro $\rho_{A}^{2}<\rho_{A}$

$\rho_{B}=Tr_{A}\rho_{AB}$ no puro $\rho_{B}^{2}<\rho_{B}$

Si uno tiene una distribución de probabilidades, la queremos caracterizar.
Una cantidad que caracteriza el grado de mezcla es la \textbf{entropía
de Shannon o Von Neumann} $S(\rho_{A})=Tr\rho_{A}\log_{2}\rho_{A}=-\sum_{k}\sigma_{k}^{2}\log_{2}\sigma_{k}^{2}$
(para 0 vale 0: $x\log x\rightarrow0$ cuando $x\rightarrow0$)

Los estados puros $S(\rho)=0\iff\rho^{2}=\rho$ (esta todo concentrado
en un solo evento (1,0,0,0))

Para estados mezcla $S(\rho)>0\iff\rho^{2}<\rho$

Dado un $\rho_{AB}$ puro $S(\rho_{AB})=0$ no hay desorden.

$\rho_{A}=Tr_{B}\rho_{AB}$ no puro entonces $S(\rho_{A})>0$

$\rho_{B}=Tr_{A}\rho_{AB}$ no puro entonces $S(\rho_{B})>0$

luego $S(\rho_{A})=S(\rho_{B})$ 

\subsection{Entropía de entrelazamiento (E)}

\[
S(\rho_{AB}=0)\rightarrow E(A,B)=S(\rho_{A})=S(\rho_{B})
\]

En todo sistema descripto por variables aleatorias (clásico)
\[
S(A,B)\geq S(A)
\]

\[
S(A,B)\geq S(A)
\]

Siendo $S(A,B)$ la entropía de la distribución de probabilidades
conjuntas. $S(AB)=\sum_{ij}p_{ij}\log_{2}p_{ij}$, $S(A)=\sum_{i}p_{i}\log p_{i}$,
$p_{i}=\sum_{j}p_{ij}$

Siempre ocurre esto, sea o no clásico. Vemos que en un estado entrelazado
cambia esto, el desorden global es menor que el desorden local.

La teoría cuántica es una nueva forma de describir sistemas aleatorio,
va más allá de la teoría de probabilidades clásicas. Porque no son
números, son operadores que no conmutan y cambian las cosas en juego.
Entonces esa entropía entrelazada existe, pero no a nivel clásico. 

\paragraph*{El estado entrelazado es una suma de estados productos. Según Schödinger
es LA mecánica cuántica. Es una nueva teoría de descripción de eventos.
Vamos a tener un marco general para describir estos eventos. Luego
vamos a explicar la física de fondo }
