\chapter{Clase 2 - 02 - 09}

\section{Entropía}

\subsection{Entropía de Von Neumann }

\[
S(\rho)=Tr\rho\log_{2}\rho=-\sum_{\nu}\EXPECT{\nu}{\nu}{\rho\log_{2}\rho}
\]


\subsection{Entropía de Shannon}

\[
S(p_{1},\dots,p_{n})=-\sum_{i}p_{i}\log_{2}p_{i}
\]
con $p_{i}\geq0$, $\sum_{i=1}^{n}p_{i}=1$

\[
\rho=\sum_{\nu}\KET{\nu}\BRA{\nu}p_{\nu}
\]

$Tr\rho=1$, $\rho^{\dagger}=\rho$, $\rho\geq0$

\[
S(\rho)=-\sum p_{\nu}\log p_{\nu}
\]
 La entropía de Von Neumann es la extensión de la de Shannon.

\subsection{Propiedades de la entropía}

\textbf{1)} $S(\rho)\geq0$

\textbf{2)} Si $S(\rho)=0$ sii $\rho^{2}=\rho$ es estado puro y
en este caso $p_{1}=1$ y el resto es 0. Ver la dem en la práctica
1. 

\textbf{3) }$S(\rho)$ es máxima sii $p_{1}=\dots=p_{n}=\frac{1}{n}$ 

La entropía es una medida de la falta de información o del grado de
mezcla.

Dado un $\rho_{AB}$ puro $S(\rho_{AB})=0$ no hay desorden.

Repasemos lo de la clase pasada:

Dado un estado conjunto, la descomposición de Schmidt

\[
\KET{\psi_{AB}}=\sum_{ij}^{d_{A},d_{B}}C_{ij}\overset{\KET{i_{A}}\otimes\KET{j_{B}}}{\KET{ij}}=\sum_{k}^{n_{S}}\sigma_{k}\KET{k_{A}k_{B}}
\]

$\sigma_{k}>0$, $\sum_{k=1}^{n_{S}}\sigma_{k}^{2}$, $n_{s}=rango(C)$

$Tr_{B}\LLAVEABAJO{\ketbra{\psi_{AB}}{\psi_{AB}}}{\rho_{AB}}=\rho_{A}=\sum_{k=1}^{n_{s}}\sigma_{k}^{2}\ketbra{k_{A}}{k_{A}}$

$Tr_{A}\LLAVEABAJO{\ketbra{\psi_{AB}}{\psi_{AB}}}{\rho_{AB}}=\rho_{B}=\sum_{k=1}^{n_{s}}\sigma_{k}^{2}\ketbra{k_{B}}{k_{B}}$

$Tr\rho_{AB}O_{AB}=\EXPECT{\psi_{AB}}{\psi_{AB}}{O_{AB}}$

Cuando el operador es un operador local

$Tr\rho_{AB}\PARENTESIS{O_{A}\otimes I_{B}}=\EXPECT{\psi_{AB}}{\psi_{AB}}{O_{A}\otimes I_{B}}=Tr\PARENTESIS{\rho_{A}O_{A}}$

Recordemos entonces que la entropía de entrelazamiento esta definida
como

\[
E(A,B)=S(\rho_{A})=S(\rho_{B})=-\sum_{k}\sigma_{k}^{2}\log_{2}\sigma_{k}^{2}
\]
 Esta entropía solamente proviene de la correlación ya que estos estados
están entrelazados.

$E(A,B)=0\iff n_{s}=1\rightarrow\text{no entrelazado (estado producto)}$

En sistemas cuánticos la entropía no es extensiva, por lo que la entropía
termodinámica (clásica o de variables aleatorias) no nos sirve. Uno
obtiene lo clásico en un límite un particular de lo cuántico.

\section{Qubits}

El sistema más simple posible, donde la $dim\mathcal{H}=2$, la base
de $\mathcal{H}$ es $\LLAVES{\KET 0,\KET 1}$

\[
\KET{\psi}=\alpha\KET 0+\beta\KET 1
\]

con $\MODULO{\alpha}^{2}+\MODULO{\beta}^{2}=1$, la forma más general
de escribir esto es:

\[
\KET{\psi}=\cos\frac{\theta}{2}\KET 0+e^{i\varphi}\sin\frac{\theta}{2}\KET 1
\]

$\rho=\ketbra{\psi}{\psi}$, un estado es un punto en la esfera de
Bloch (esfera de radio 1)

Dada $\sigma_{z}=\PARENTESIS{\begin{array}{cc}
1 & 0\\
0 & -1
\end{array}}$, $\sigma_{z}\KET 0=1\KET 0$, $\sigma_{z}\KET 1=-1\KET 1$

$\sigma_{x}=\PARENTESIS{\begin{array}{cc}
0 & 1\\
1 & 0
\end{array}},\sigma_{y}=\PARENTESIS{\begin{array}{cc}
0 & 1\\
-1 & 0
\end{array}}$, $\sigma_{0}=\PARENTESIS{\begin{array}{cc}
1 & 0\\
0 & 1
\end{array}}$

$S_{z}=\frac{\hbar}{2}\sigma_{z}$, $\vec{S}=\frac{\hbar}{2}\PARENTESIS{\sigma_{x},\sigma_{y},\sigma_{z}}$

$\EXPECT{\psi}{\psi}I=1$, $\EXPECT{\psi}{\psi}{\sigma_{z}}=\cos\theta$,
$\EXPECT{\psi}{\psi}{\sigma_{x}}=\sin\theta\cos\varphi$, $\EXPECT{\psi}{\psi}{\sigma_{y}}=\sin\theta\sin\varphi$
\begin{center}
\includegraphics{pegado32}
\par\end{center}

\subsection{Estado general de un qubit}

Estado puro $\KET{\psi}=\cos\frac{\theta}{2}\KET 0+e^{i\varphi}\sin\frac{\theta}{2}\KET 1$

Estado no puro: $\rho=p\ketbra{\psi_{1}}{\psi_{1}}+(1-p)\ketbra{\psi_{2}}{\psi_{2}}$

En el caso que $\BRAKETEO{\psi_{1}}{\psi_{1}}=0\rightarrow$ $\KET{\psi_{1}},\KET{\psi_{2}}$son
autovectores, $(p,1-p)$ son autovalores de $\rho$. Si no son ortogonales
es un estado mezcla.

$\rho=\alpha\sigma_{0}+\frac{1}{2}\PARENTESIS{r_{x}\sigma_{x}+r_{y}\sigma_{y}+r_{z}\sigma_{z}}$

$Tr(\rho)=1=\alpha TrI+0=2\alpha\rightarrow\alpha=1/2$

\[
Tr(\rho\sigma_{\mu})=\frac{1}{2}2r_{\mu}=\VALMEDIO{\sigma_{\mu}}\rightarrow r_{\mu}=\VALMEDIO{\sigma_{\mu}}
\]

\[
\rho=\frac{1}{2}\PARENTESIS{I+\vec{r}\cdot\vec{\sigma}}\rightarrow\vec{r}=\VALMEDIO{\vec{\sigma}},\MODULO{\vec{r}}\leq1
\]
Un estado mixto en vez de ser un punto sobre la esfera es un punto
dentro de la esfera.

Punto en la esfera: estado puro

Punto dentro de la esfera: estado mezcla

Punto en el centro: estado completamente mezclado. $\rho=\frac{1}{2}I,\:\MODULO{\vec{r}}=0$

La flecha del origen al punto es $\vec{r}$.

\[
\MODULO{\vec{r}}=1\iff\rho\text{ es puro}
\]

\[
0\leq\MODULO{\vec{r}}<1\iff\rho\text{ no puro}
\]

Definamos la dirección $z'\parallel\vec{r}$, entonces $\rho=\frac{1}{2}\PARENTESIS{I+\vec{r}\cdot\vec{\sigma}}=\frac{1}{2}\PARENTESIS{I+\MODULO{\vec{r}}\sigma_{z'}}$.
De aquí: los autovalores de $\rho$ son

$\lambda(\rho)=\frac{1}{2}\PARENTESIS{1\pm\MODULO{\vec{r}}}=\lambda_{\pm}$

$\rho=\lambda_{+}\ketbra{0'}{0'}+\lambda_{-}\ketbra{1'}{1'}$

La polarización de un fotón es una dimensión de un qubit.

\subsection{No cloning theorem}

¿Qué sería una fotocopiadora cuántica?

\[
U\PARENTESIS{\KET{\psi}\LLAVEABAJO{\KET 0}{\text{est. en la hoja a copiar}}}=\KET{\psi}\KET{\psi}\qquad\forall\KET{\psi}
\]

No existe ese operador $U$. Supongamos que si existe y pude crear
otro estado

\[
U\PARENTESIS{\KET{\varphi}\KET 0}=\KET{\varphi}\KET{\varphi}
\]

\[
\PARENTESIS{\BRA{\varphi}\BRA 0}\LLAVEABAJO{U^{\dagger}U}I\PARENTESIS{\KET{\psi}\KET 0}=\BRAKETEO{\varphi}{\psi}\BRAKETEO{\varphi}{\psi}=\BRAKETEO{\varphi}{\psi}^{2}
\]

Entonces $\BRAKETEO{\varphi}{\psi}=\BRAKETEO{\varphi}{\psi}^{2}\rightarrow\BRAKETEO{\varphi}{\psi}=0\;o\;\BRAKETEO{\varphi}{\psi}=1$

Vemos que podemos diseñar $U$ solo que copie estados ortogonales,
pero no sabemos cómo es el estado que va a venir.

\subsection{Representación gráfica de un estado de un qubit}

\[
\underset{t=t_{0}}{\KET{\psi}}\rightarrow\overset{U}{\APLbox}\rightarrow U\underset{t=t_{1}}{\KET{\psi}}
\]

$U=U(t_{1},t_{0})$

Ahora para 2 qubits

\[
\left.\begin{array}{c}
\underset{}{\KET{\psi_{A}}}\rightarrow\overset{U_{A}}{\APLbox}\rightarrow U_{A}\underset{}{\KET{\psi_{A}}}\\
\underset{}{\KET{\psi_{B}}}\rightarrow\overset{U_{B}}{\APLbox}\rightarrow U_{A}\underset{}{\KET{\psi_{B}}}
\end{array}\right\} U_{A}\KET{\psi_{A}}U_{B}\psi_{B}
\]

Ahora si parto de un estado producto 
\[
\KET{\psi_{AB}}=\KET{\psi_{A}}\KET{\psi_{B}}
\]
y el operador unitario $U_{AB}=U_{A}\otimes U_{B}$
\[
U_{AB}\KET{\psi_{AB}}=U_{A}\KET{\psi_{A}}U_{B}\psi_{B}
\]

$U_{A}=e^{-i\frac{H_{A}t}{\hbar}},\quad U_{B}=e^{-i\frac{H_{B}t}{\hbar}}$
, con $t=t_{1}-t_{0}$

Este es un sistema sin interacción, el $H=H_{A}+H_{B}$, este tipo
de hamiltoniano no genera entrelazamiento, un estado producto evoluciona
como estado producto.

\textbf{Otro ejemplo, de un estado no producto:}

\[
\KET{\psi_{AB}}=\frac{\KET{00}+\KET{11}}{\sqrt{2}}
\]
Vemos que ya es un estado en la forma de Schmidt. Este es un estado
entrelazado. 

En cambio el estado $\KET{\psi_{AB}}=\frac{\KET{00}+\KET{01}-\KET{10}-\KET{11}}{2}$
a simple vista no parece estado producto pero rescribamoslo (sacando
factor común al $\KET 0$ y al $\KET 1$):

\[
\frac{\KET 0-\KET 1}{\sqrt{2}}\otimes\frac{\KET 0+\KET 1}{\sqrt{2}}
\]
 vemos que en realidad es un estado producto.

\subsection{Bases:}

\subsubsection{Base computacional:}

\[
\LLAVES{\KET{00},\KET{01},\KET{10},\KET{11}}
\]


\subsubsection{Base de Bell: formada por estados entrelazados:}

$\LLAVES{\frac{\KET{00}\pm\KET{11}}{\sqrt{2}},\frac{\KET{01}\pm\KET{10}}{\sqrt{2}}}$
(estos estados ya estan en la forma de Schmidt. 

En información cuántica nos interesa tener en la base de la forma
de compuertas lógicas.

\section{Compuertas lógicas:}

Ver la figura del primer circuito cuántico, que me genera un estado
entrelazado

\subsection{Hadamard}

\[
H=U_{H}=\frac{1}{\sqrt{2}}\PARENTESIS{\begin{array}{cc}
1 & 1\\
1 & -1
\end{array}}=\frac{1}{\sqrt{2}}\PARENTESIS{\sigma_{z}+\sigma_{x}}
\]

\[
U_{H}\KET 0=\frac{\KET 0+\KET 1}{\sqrt{2}}
\]

\[
U_{H}\KET 1=\frac{\KET 0-\KET 1}{\sqrt{2}}
\]

Me hace la mezcla de estados computacionales en estados equiprobables.

\subsection*{Control not}

\[
U_{CNOT}\KET{00}=\KET{00}
\]

\[
U_{CNOT}\KET{01}=\KET{01}
\]

\[
U_{CNOT}\KET{10}=\KET{11}
\]

\[
U_{CNOT}\KET{11}=\KET{10}
\]

Estas compuertas son completas, esto quiere decir que con cualquiera
de ellas puedo generar cualquier operador unitario. Estos operadores
puedo hacer cualquier tranformación unitaria para cualquier cantidad
de qubits. 

Si uno va cambiando con los estados que entra a estos circuitos voy
a ir obteniendo los diferentes estados de Bell.
