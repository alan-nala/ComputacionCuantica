\chapter{Entrelazamiento Cuántico}

\begin{center}
  \begin{minipage}{0.5\textwidth}
    \begin{small}
      `` Solo sé que no sé nada y, al saber que nada sé, algo sé.'' \par
%\setlength{\parindent}{30ex}
 \begin{flushright}{\it Sócrates}
 \end{flushright}
    \end{small}
\end{minipage}
  \vspace{0.5cm}
\end{center}

Una propiedad clave que caracteriza el comportamiento de los sistemas en el régimen cuántico es el llamado 
\emph{Entrelazamiento cuántico}. Esta propiedad puede entenderse como la imposibilidad de representar las correlaciones existentes en un sistema
cuántico en términos de una distribución estadística sobre posibles configuraciones del sistema, especificables en términos de estados locales definidos.
Para entender cómo esta noción es fundamental en la descripción clásica de los sistemas físicos, consideremos por ejemplo el sistema Tierra-Luna-Sol.
Para la mecánica newtoniana, estos tres astros son entes distintos, con propiedades independientes. Así, en cada instante el Sol, la Tierra y la Luna 
cuentan con posiciones y velocidades bien definidas. El efecto de la interacción entre estos cuerpos se reduce entonces a cambiar las velocidades de 
estos cuerpos en función de las posiciones de los otros. Este marco de trabajo, que podemos llamar \emph{reduccionista}, permite describir y predecir
el comportamiento de la mayoría de los sistemas físicos macroscópicos en forma a la vez computacionalmente eficiente y precisa. Sin embargo, al tratar
de aplicarlo a sistemas en la escala atómica comienza a mostrar sus fallas: en esta escala, el entrelazamiento implica que para ciertos estados 
del sistema, las propiedades de las partes no estén bien definidas. Por ejemplo, el estado típico de una molécula de Hidrógeno no puede en general 
describirse en términos de los estados de los átomos que la conforman: existen observables globales (la energía de la molécula, el impulso angular) 
que no son  compatibles con los observables asociados a los átomos por separado. Esto nos obliga en mecánica cuántica a tratar los sistemas compuestos 
desde una perspectiva \emph{holística}, en el sentido de que debemos tratar al sistema como un todo. Una consecuencia inmediata es que la descripción exacta 
de los sistemas cuánticos se vuelve mucho más compleja que la de su equivalente clásico.

Sin embargo, no todos los sistemas cuánticos parecen requerir de una descripción que aborde toda esa complejidad potencial. Por ejemplo, el sistema 
Tierra-Sol-Luna en sí es un sistema cuántico. La clave del éxito de su descripción clásica consiste en que en su evolución el estado de cada una de
sus partes permanece bien definido. Decimos entonces que el sistema admite descripción en términos de  \emph{estados separables}. Por otro lado, la 
molécula de Hidrógeno típicamente se encuentra en un estado en el que el estado de sus partes no está bien definido. Decimos por esto que su descripción
requiere considerar \emph{estados entrelazados}.  Podemos decir entonces que cuanto más entrelazados se encuentren los estados típicos de un sistema 
(en un dado régimen) más nos costará representarlos en forma precisa y eficiente a la vez.

%Si bien la definición de entrelazamiento no parece ofrecer grandes dificultades, 
La definición y evaluación
de medidas que cuantifiquen el grado de entrelazamiento presente en un sistema es una tarea bastante más compleja, tanto desde un punto de vista formal 
como práctico. En lo que resta del capítulo daremos una definición más formal de lo que es un  \emph{estado cuántico entrelazado} \cite{NC.00,V.07,Sch1,BDSW.96}. 
Se introducirán entonces algunos elementos de la teoría de la información que nos permitirán definir medidas de entrelazamiento y de correlaciones cuánticas entre
 partes de un sistema.
 
%  En los capítulos siguientes, utilizaremos estas cantidades para estudiar algunos aspectos del comportamiento cadenas de espines en el régimen cuántico, 
% % así como para analizar la aplicabilidad de la aproximación de \emph{Campo Medio Generalizado} para su descripción.

\section{Operador Densidad y Entropía de Von Neumann}
%Nos interesa ahora extender las definiciones previas a sistemas cuánticos
%\cite{NC.00,V.07}.  
%nte destacar desde el principio que no todas las
%extensiones posibles son necesariamente triviales, y que no todas las desigualdades anteriores
%continuarán siendo v\'alidas.
El estado de un sistema cuántico 
%que cuyos estados pertenecen a un espacio de Hilbert de dimensión finita $n$
se puede caracterizar por un {\it operador densidad} (o
matriz densidad) $\rho$, el cual es un operador hermítico de traza 1 y con todos sus autovalores no negativos:
\begin{equation}\rho\geq 0\,,\;\;{\mathrm Tr}\rho=1\,.\end{equation}
Este operador determina el valor medio de cualquier observable $O$: 
% se puede determinar de la formaexpresi\'on
%Recordemos que el valor medio de cualquier observable $O$ est\'a dado por
\begin{equation}\langle O\rangle={\rm Tr}\,\rho\,O\,.\end{equation}
La probabilidad de encontrar al sistema en un estado particular $|i\rangle$,
(que supondremos normalizado) es entonces
\begin{equation}
p_i=\langle P_i\rangle={\rm Tr}\,\rho\,P_i=\langle i|\rho|i\rangle\,,
\end{equation}
donde $P_i=|i\rangle \langle i|$ es el proyector ortogonal sobre el estado $|i\rangle$.
%Recordemos tambi\'en que si la descomposici\'on espectral de $O$ es $O=\sum_\nu    o_\nu|\nu\rangle\langle\nu|$,
%la probabilidad de medir el valor $o_\nu$ ser\'a, asumiendo este no degenerado,
%\[p_{\nu}=\langle \nu|\rho|\nu\rangle\,.\]
%En el caso general,
%\[p_{o_\nu}={\rm Tr}\rho\,P_{p_\nu}\]
%siendo $P_{o_\nu}$ el proyector ortogonal sobre el autoespacio asociado. % al autovalor $o_\nu$.

En el caso de un sistema cuántico en un estado puro
$|i\rangle$, $\rho$ es entonces el proyector ortogonal sobre el espacio generado por $|i\rangle$:
\begin{equation}\rho=|i\rangle\langle i|\,,
\end{equation}
y satisface $\rho^2=\rho$. En el caso general, la descomposición espectral de $\rho$ la escribiremos como
%$\rho^2\leq\rho$ (es decir,
%$\rho-\rho^2$ es un operador positivo).
\begin{equation}\rho=\sum_i p_i|i\rangle\langle i|\,,\end{equation}
donde $\{p_i,i=1,\ldots,n\}$ %con $n$ la dimensi\'on del espacio,
son los autovalores de $\rho$ ($p_i\geq 0$, $\sum_i p_i=1$) y $\{|i\rangle,i=1,\ldots,n\}$ los
correspondientes autovectores normalizados ($\langle i|i'\rangle=\delta_{ii'}$). El caso puro corresponde a $p_i=1$ para un cierto estado y 0 para todos los demás.
En el caso general, tenemos $\rho^2\leq \rho$ (es decir, $\rho^2-\rho$ es un
operador con autovalores $-p_i(1-p_i)$ negativos o nulos).

%Recordemos que el valor medio de cualquier observable $O$ esta dado por
La entropía de von Neumann\cite{NC.00,Wh,vn.27} se define como
\begin{eqnarray}S(\rho)&=&-{\rm Tr}\,\rho\log\rho\\&=&
-\sum_i p_i\log p_i\,,\end{eqnarray}
y es una medida de la falta de información asociada al estado $\rho$. Tenemos $S(\rho)\geq 0$, con $S(\rho)=0$ únicamente 
si $\rho$ es un estado puro ($\rho^2=\rho$).
$S(\rho)$ será por el contrario máxima ($S(\rho)=\log n$ si el espacio de Hilbert del sistema tiene dimensión $n$) si el estado $\rho$ es máximamente ``mezclado''
$\rho_n=I_n/n$, donde $I_n$ denota el operador identidad, tal que $p_i=1/n$ $\forall$ $i$.

\section{Sistemas Compuestos y Estados Reducidos}

Dados dos sistemas cuánticos distinguibles, que denotaremos como $A$ y $B$, con sendos espacios de Hilbert $\mathcal{H}_A$ y $\mathcal{H}_B$ 
y espacio de Hilbert conjunto $\mathcal{H}_A\otimes \mathcal{H}_B$, el estado conjunto estará determinado por una cierta {\it 
matriz densidad conjunta} $\rho_{AB}$. La entropía conjunta es por lo tanto
\begin{equation}S(A,B)=S(\rho_{AB})=-{\rm Tr}\,\rho_{AB}\log \rho_{AB}\,.
 \label{SAB}\end{equation}
%Tenemos en mente, como caso t\'{\i}pico, dos sistemas distinguibles $A$, $B$
%espacialmente separados,  aunque las siguientes expresiones son v\'alidas para
%todo par $A$, $B$ con espacio conjunto accesible $H_A\otimes H_B$. 
Un observable {\it local} en el sistema $A$ es un observable de la forma
$O_A\equiv O_A\otimes I_B$, donde $I_B$ denota la identidad en $\mathcal{H}_B$. Su valor
medio es entonces
\begin{eqnarray}
\langle O_A\rangle&=&{\rm Tr}\,\rho_{AB}O_A%\nonumber\\
={\rm Tr}_A\, \rho_A O_A\,,
\end{eqnarray}
donde hemos definido la {\it matriz densidad reducida} \cite{NC.00,Wh}
\begin{equation}
 \rho_A={\rm Tr}_B\,\rho_{AB}\label{rhoa}\,\end{equation}
la cual determina completamente los valores medios de todo observable local en $A$.
Explícitamente, $\langle i|\rho_A|j\rangle=\sum_k\langle ik|
\rho_{AB}|jk\rangle$, donde $|ik\rangle\equiv|i\rangle\otimes|k\rangle$ son
los estados de una base producto ortonormal de $\mathcal{H}_A\otimes \mathcal{H}_B$. Análogamente,
\[\rho_B={\rm Tr}_A\,\rho_{AB}\,,\]
determina los valores medios de cualquier operador local $O_B$ en $B$. Las
entropías locales son
\[S(A)=-{\rm Tr}\,\rho_A\log\rho_A\,,\;\;\;
 S(B)=-{\rm Tr}\,\rho_B\log\rho_B\,.\]
El estado conjunto es no correlacionado si y solo si $\rho_{AB}=\rho_A\otimes
\rho_B$, es decir, si y solo si es un estado producto, en cuyo caso
sus autovalores son $p_{ij}=p_i^A p_j^B$ con $p_i^A$ y $p_j^B$ 
los autovalores de $\rho_A$ y $\rho_B$ respectivamente.  
%es decir, si est\'a completamente determinado por los operadores
%densidad reducidos (o sea, por observaciones locales), en cuyo caso 
%Esto corresponde al caso
%de  v.a.\ independientes en sistemas clásicos.
En tal caso las entropías
satisfacen $S(A,B)=S(A)+S(B)$, como es fácil ver de la
definición (\ref{SAB}).
\section{Información  Mutua y Entropía Condicional}

%Podemos definir ahora la {\bf entropía condicional} utilizando la %f\'ormula\cite{NC.00,Wh,OZ.01}:
%\begin{eqnarray}S_1(A|B)&\equiv&S(A,B)-S(B)\label{qcond1}\\
% &=&S(\rho_{AB})-S(\rho_B)\,.\label{qcond11}\end{eqnarray}
%No obstante, es fundamental notar que esta cantidad {\it 
%no es necesariamente positiva}. Es decir, la entrop\'{\i}a global
%$S(A,B)$ puede ser {\it mayor}, pero también {\it menor} que las entrop\'{\i}as locales.
%El caso menor se da en los estados {\it entrelazados}, que definiremos en detalle
%en el pr\'oximo cap\'{\i}tulo, y del cual se dar\'a  un ejemplo en este cap\'{\i}tulo.

%La segunda versión cuántica de la entropía condicional fue introducida por
%Zurek en 2001 \cite{Zu1.01} y la discutiremos en el cap\'{\i}tulo 4 cuando describamos el {\it Quantum Discord}.
% ({\it Discordancia Cuántica}).

%En base a $S_1(X|Y)$, podemos definir la primera versión c\'uantica de
Podemos ahora definir la {\bf información mutua} como % siguiendo las ecuaciones (\ref{mut1})--(\ref{mut2}):
\begin{eqnarray}
I(A:B)&=&S(\rho_A)+S(\rho_B)-S(\rho_{AB})\,.\label{qmut1}%\\
%&=&S(A)-S_1(A|B)=S(B)-S_1(B|A)\label{qmut0}\\
%&=&S(A)+S(B)-S(A,B)
\end{eqnarray}
Esta cantidad es una medida de la correlación (total) entre $A$ y $B$
\cite{NC.00,Wh}. Si $\rho_{AB}=\rho_A\otimes\rho_B$, %$A$ y $B$ son no correlacionados,
$S(\rho_{AB})=S(\rho_A)+S(\rho_B)$ y por lo tanto $I(A:B)=0$. En caso contrario 
$I(A:B)>0$. 

%A diferencia de $S_1(A|B)$, $S_1(A:B)$ es {\it positiva}: $S_1(A:B)\geq 0$. % tal como se demuestra en el ap\'endice. 
Esta positividad de $I(A:B)$ es conceptualmente evidente: $S(A)+S(B)$ es una medida de la falta de
información cuando s\'olo se dispone de información sobre los valores medios de todos los observables locales (es decir, cuando se conoce s\'olo $\rho_A$ y $\rho_B$), 
 mientras que $S(A,B)$ mide la falta de información cuando se conoce además
toda la información sobre las correlaciones,
es decir, sobre todos los valores medios de observables generales del tipo $O_{AB}=O_A\otimes O_B$. Por lo tanto $S(A,B)\leq S(A)+S(B)$.

Clásicamente, es decir, para sistemas descriptos por densidades de probabilidad, se tiene además 
\begin{equation}
S(A,B)\geq S(A),\;\;\;S(A,B)\geq S(B)\label{desc}\end{equation}
Las entropías condicionales $S(A|B)$ y $S(B|A)$ pueden definirse como 
\begin{equation}S(A|B)=S(A,B)-S(B),\;\;S(B|A)=S(A,B)-S(A)\label{Scond}\end{equation}
y son por lo tanto cantidades no negativas en sistemas clásicos. 
%Podemos as\'{\i} reescribir la informaci\'on mutua como 

Sin embargo, la desigualdad (\ref{desc}) no sigue siendo válida en sistemas cuánticos, es decir, en sistemas descriptos por operadores densidad. En otras palabras, en sistemas cuánticos la entropía global puede ser menor que las entropías locales, y las entropías condicionales definidas como en (\ref{Scond}) pueden por lo tanto ser negativas. 


A modo de ejemplo, consideremos un par de qubits o espines $1/2$
en un estado de Bell, por ejemplo
\begin{equation}|\Psi\rangle=\frac{|\!\!\uparrow\uparrow\rangle+
|\!\!\downarrow\downarrow\rangle}{\sqrt{2}}
 \label{bell1}\,,\end{equation}
donde $|\!\!\uparrow\uparrow\rangle=|\!\!\uparrow\rangle\otimes|\!\!\uparrow\rangle$
denota un estado con ambos espines en la dirección $z$ positiva. 
%Este estado
%lo reescribiremos, siguiendo la notación acostumbrada en información cuántica,
%como
% \begin{equation}|\Psi\rangle=\frac{|00\rangle+|11\rangle}{\sqrt{2}}\,.
% \label{bell2}\end{equation}
El estado $\rho=|\Psi\rangle\langle\Psi|$ es un estado puro y por lo tanto, 
%Para este estado, obviamente 
\[S(\rho_{AB})=0\,.\] 
%(pues $\rho_{AB}=|\Psi\rangle\langle\Psi|$
%orresponde a un estado puro ($\rho_{AB}^2=\rho_{AB}$))
No obstante, los estados reducidos son máximamente mezclados: 
\[\rho_A=\rho_B=\frac{1}{2}I_2=\frac{1}{2}(|\uparrow\rangle\langle \uparrow|+|\downarrow\rangle\langle \downarrow|)\,.\]
%es decir, las matrices densidad reducidas est\'an m\'aximamente mezcladas
Por lo tanto,
\begin{equation}
S(\rho_A)=S(\rho_B)=1\,,\end{equation}
tomando el logaritmo en base $2$.  Esto implica $S(A)=S(B)>S(A,B)=0$, a diferencia de cualquier sistema clásico. M\'as aun, los estados locales están máximamente mezclados (es decir, máximamente ``desordenados'') a pesar de  que el estado global es puro (es decir, máximamente ``ordenado''). Para este estado tenemos entonces 
\[I(A:B)=2\]
\[S(A|B)=S(B|A)=-1\,.\]
Como veremos a continuación, la violación de las desigualdades clásicas (\ref{desc}) puede darse solo cuando el estado $\rho$ es entrelazado. 

%Este estado es un estado máximamente entrelazado de dos qubits, en el que el
%sistema conjunto está ``máximamente ordenado'' (está en un estado puro) pero el
%sistema local está máximamente desordenado. %Esto es imposible en el caso cl\'asico, ya que si $p_{ij}=1$ para un cierto par $(i,j)$ (y 0 para los restantes) entonces
%necesariamente $p^A_i=p^B_j=1$ y entonces $S(A)=S(B)=0$.
%Este estado conduce pues a un caso
%extremo de no extensividad: Mientras que $S(A)=S(B)=1$, la entropía global $S(A,B)$, lejos de ser
%igual a la suma, es {\bf nula}.

%Puede tambi\'en afirmarse que el estado de Bell corresponde a un caso de correlaci\'on extremo,
%sin an\'alogo cl\'asico, en el que
%\[S_1(A:B)=S(A)+S(B)-0=2\]
%En el caso de dos sistemas cl\'asicos binarios, el valor m\'aximo de $S(A:B)$ es 1 y
%corresponde a $p_{ij}=\delta_{ij}/2$, es decir a un sistema en el que ambos estan o bien
%en el estado $0$, o bien en el estado $1$, con igual probabilidad.

%Finalmente, mencionemos que todo el formalismo estadístico clásico se recupera
%como caso particular del cuántico si nos restringimos a operadores densidad
%conjuntos de la forma
%\begin{equation}\rho=\sum_{i,j}p_{ij}|ij\rangle\langle ij|
%\label{rhoclas}\end{equation}
%es decir, diagonales en una base
%producto $\{|ij\rangle=|i\rangle\otimes|j\rangle$\},
%y a observables que son diagonales en esta base (y por lo tanto conmutantes y simult\'aneamente medibles).
%En este caso,
%las matrices densidad reducidas son $\rho_A=\sum_{i}p_i^A|i\rangle\langle i|$,
%$\rho_B=\sum_j p_j^B|j\rangle\langle j|$ y quedan completamente determinadas
%por las distribuciones marginales $p_i^A=\sum_j p_{ij}$,
%$p_j^B=\sum_i p_{ij}$, como en el primer cap\'{\i}tulo. Los valores medios de
%observables diagonales $O_{AB}$, $O_A$ y $O_B$ son pues determinados puramente
%por la distribuciones $p_{ij}$, $p^A_{i}$ y $p^B_{i}$
%respectivamente.

%\section{Apéndice} Demostremos que la entropía mutua cuántica es también positiva
%\cite{Wh}:
%\[S_1(A:B)\geq 0\,,\]
%con $S_1(A:B)=0$ si y solo si $\rho_{AB}=\rho_A\otimes\rho_B$. \\
%Utilizando la misma propiedad anterior del logaritmo, y escribiendo
%\[\rho_{AB}=\sum_\nu p_\nu|\nu\rangle\langle\nu|,  \;\;
%\rho_A%={\rm Tr}_B\,\rho_{AB}
%=\sum_{i} p_i^A|i\rangle\langle i|\,,\;\;
%\rho_B%={\rm Tr}_A\,\rho_{AB}
%=\sum_{j} p_j^B|j\rangle\langle j|\,,\]
%con $p_i^A=\sum_{\nu,j}p_\nu|\langle ij|\nu\rangle|^2\,,
%\;\;p_j^B=\sum_{\nu,i}p_\nu|\langle ij|\nu\rangle|^2\,,$
%y $|ij\rangle=|i\rangle\otimes |j\rangle$ los estados de una base producto
%ortonormal formada por los autoestados de $\rho_A$ y $\rho_B$, se tiene,
%utilizando nuevamente  que
% $\log x\leq x-1$,
%\[S_1(A:B)=-\sum_{i,j,\nu}p_\nu|\langle ij|\nu\rangle|^2\log\frac{p_i^A p_j^B}{p_\nu}
%\geq \sum_{i,j,\nu}p_\nu|\langle ij|\nu\rangle|^2(\frac{p_i^A p_j^B}{p_\nu}-1)
% =1-1=0\]
%donde la igualdad se cumple solo si $p_i^A p_j^B=p_\nu$ para algún $\nu$
%$\forall$  $i,j$, es decir si $p_\nu=p_i^A\,p_j^B$ con
%$|\nu\rangle=|ij\rangle$.

\section{Entrelazamiento de Estados Puros}
Si un estado cuántico puro $|\Psi_{AB}\rangle$ de un sistema conjunto $A+B$ se puede escribir como
estado producto, no posee entrelazamiento y se lo denomina  {\it separable}. Por otro lado un estado {\it entrelazado} no puede descomponerse en un producto de estados: 
\begin{eqnarray}|\Psi_{AB}\rangle=|\Psi_A\rangle|\Psi_B\rangle&\Rightarrow&|\Psi_{AB}\rangle\;separable\\
 |\Psi_{AB}\rangle\neq |\Psi_A\rangle|\Psi_B\rangle&\Rightarrow&|\Psi_{AB}\rangle\; entrelazado
 \label{QE1}\end{eqnarray}
Las entropías de los subsistemas de un estado puro son idénticas (v\'ease (\ref{rred}) y permiten definir la {\it entropía de entrelazamiento} \cite{NC.00,Sch1},  que cuantifica el {\bf entrelazamiento} de un estado cuántico puro bipartito, como 
\begin{equation}E(A,B)=S(A)=S(B)\,.\label{EAB}\end{equation}
$E(A,B)$ es una medida de las correlaciones
cuánticas en el estado. Si $|\Psi_{AB}\rangle$ es separable, entonces $\rho_A=|\Psi_A\rangle\langle
\Psi_A|$, $\rho_B=|\Psi_B\rangle\langle\Psi_B|$ y $E(A,B)=0$.

En el caso puro $E(A,B)$ es menos la entropía condicional:
\begin{equation}S(A|B)=S(B|A)=-E(A,B)\,,\end{equation}
pues $S(A,B)=0$. Mientras que la correspondiente información mutua es
\begin{equation}I(A:B)=S(A)-S(A|B)=2S(A)=2E(A,B)\,.\end{equation}
Podemos considerar a $I(A:B)$ como una medida de todas las correlaciones
en el sistema, mientras que a $E(A,B)$ como una medida de correlaciones
puramente cuánticas.

Una forma de determinar si un estado cuántico es entrelazado es a través de la
{\it descomposición de Schmidt} del estado \cite{NC.00}:
Existen siempre bases locales $\{|k_A\rangle\}$ y $\{|k_B\rangle\}$
ortonormales, en las que $|\Psi\rangle$ puede escribirse en la forma
\begin{equation}|\Psi\rangle=\sum_{k=1}^{n_s} \sigma_k|k_A\rangle|k_B\rangle
\,,\label{SD}
\end{equation}
donde $n_s$ es el número de Schmidt y $\sigma_k>0$,
$\sum_{k=1}^{n_s}\sigma_k^2=1$. Las matrices densidad
reducidas están entonces dadas por
\begin{equation}\rho_A=\sum_{k} \sigma_k^2|k_A\rangle\langle k_A|\,,\;\;
\rho_B=\sum_{k} \sigma_k^2|k_B\rangle\langle k_B|\label{rred}\,.\end{equation}
Estas son isospectrales por lo que $S(A)=S(B)$.
%\begin{eqnarray}\rho_A&=&{\rm Tr}_B |\Psi\rangle\langle\Psi|
%=\sum_{k} \sigma_k^2|k_A\rangle\langle k_A|,\\
%\rho_B&=&{\rm Tr}_A |\Psi\rangle\langle\Psi|
% =\sum_{k}\sigma_k^2|k_B\rangle\langle k_B|\,,\end{eqnarray}
%siendo pues estas bases locales (denominadas bases de Schmidt) justamente las
%que diagonalizan los operadores densidad reducidos. Estos son pues siempre
%isospectrales, implicando $S(A)=S(B)$.
El caso separable corresponde a $n_s=1$,
donde  $E(A,B)=0$, mientras que el caso entrelazado a $n_s\geq 2$, en el que
\begin{equation}E(A,B)=-\sum_{k=1}^{n_s}\sigma_k^2\log (\sigma_k^2)
\label{EAB2}
\end{equation}
La descomposición de Schmidt puede obtenerse a partir de la
{\itshape descomposición en valores singulares} de la matriz de los coeficientes
de expansión de $|\Psi\rangle$ en una base producto ortogonal, arbitraria
\cite{NC.00}, siendo los $\sigma_k$ los valores singulares de dicha matriz.
En el caso del estado de Bell (ecuación \ref{bell1}), ya está
expresado en una base de Schmidt, con $n_s=2$ y
$\sigma_1=\sigma_2=1/\sqrt{2}$. 

El entrelazamiento es considerado un {\it recurso
esencial} en información cuántica \cite{NC.00,V.07}, ya que permite formas
radicalmente nuevas de intercambio y procesamiento de la información,
tales como la teleportación cuántica \cite{t.93} y la computación cuántica \cite{NC.00}. 
% imposibles de simular eficientemente en sistemas clásicos.

\section{Entrelazamiento de Estados no Puros}
La definición de entrelazamiento cuántico es m\'as compleja para estados $\rho$ generales 
no necesariamente puros ($\rho^2\leq\rho$). De hecho, en el caso general no es posible obtener un m\'etodo general 
para determinar si el estado es entrelazado, en un n\'umero finito de pasos. Por lo tanto tampoco es posible
obtener una medida computable del mismo.

%entrelazamiento, e incluso, no est\'a a\'un resuelto el problema formal
%de definir una medida de todas las correlaciones cuánticas en estados no puros.

Según la definición introducida por R.F. Werner en 1989 \cite{W.89}, un estado
cuántico general es {\it entrelazado} si no es {\it separable} o {\it clásicamente correlacionado}, 
en cuyo caso puede ser escrito como una combinación convexa de estados producto, es decir,
una superposición estadística de estados no correlacionados:
\begin{eqnarray}\rho=\sum_{\alpha} q_\alpha \rho_A^\alpha\otimes\rho_B^\alpha,
\;\;q_\alpha\geq 0,\;&\Rightarrow & \rho\;separable\label{rs2}\\
\rho\neq \sum_{\alpha} q_\alpha \rho_A^\alpha\otimes\rho_B^\alpha,
 \;\;q_\alpha\geq 0,&\Rightarrow& \rho\;entrelazado\end{eqnarray}
donde $\sum_\alpha q_\alpha=1$. En particular, un estado producto
$\rho_{AB}=\rho_A\otimes\rho_B$, es decir, un estado no correlacionado, es un
estado separable. Pero también lo es cualquier combinación convexa de los
mismos. El argumento \cite{W.89} es que los estados separables pueden ser
generados mediante operaciones locales y comunicación clásica (es decir, por
({\it LOCC: Local Operations and Classical Communication}) \cite{NC.00}) y por lo tanto no
contienen correlaciones cuánticas.
`
En otras palabras, dos personas a cierta distancia pueden, a través de comunicación clásica, acordar
preparar un estado producto $|\Psi_A\rangle|\Psi_B\rangle$, pero también una combinación
estadística de estados producto: $A$ tira un dado y de acuerdo al valor de este
prepara $|\Psi^\alpha_A\rangle$, $\alpha=1,\ldots,6$ y avisa a $B$, quien
prepara el correspondiente estado $|\Psi^\alpha_B\rangle$, originando as\'{\i}
una combinación convexa del tipo (\ref{rs2})
($\rho=\sum_{\alpha=1}^6\frac{1}{6}|\Psi_A^\alpha\rangle\langle\Psi_A^\alpha|\otimes|\Psi_B^\alpha\rangle
\langle\Psi_B^\alpha|$).

Por otro lado un estado entrelazado no puede ser
escrito de la forma anterior con coeficientes $q_\alpha$ positivos. Estos se generan únicamente por medio de una 
interacción cuántica entre los sistemas. Pueden generarse como autoestados de un Hamiltoniano que contenga t\'erminos de interacción $\sum_\alpha o^\alpha_{A}\otimes o^\alpha_B$, o haciendo evolucionar 
un estado inicialmente separable con un Hamiltoniano del tipo anterior \cite{NC.00,RS.07} (de forma que el operador evolución 
$U(t)=\exp[-iH t/\hbar]$ no sea un producto de operadores de evolución
locales $U_A(t)\otimes U_B(t)$ ).
%  o sencillamente como estados cu\'anticas, es decir, por medio de

Los estados $\rho$ diagonales en una base producto: $\rho=\sum_{i,j}p_{ij}|ij\rangle\langle ij|
\label{rhoclas}$ 
% (discutidos en el cap\'{\i}tulo anterior) 
son un caso particular de estado separable.
En el caso general, los distintos t\'erminos en (\ref{rs2}) no son necesariamente conmutantes.

En el caso puro, la definición (\ref{rs2}) coincide por supuesto con la previa
dada en la ecuación (\ref{QE1}): Si $\rho_{AB}^2=\rho_{AB}$, la combinación convexa
(\ref{rs2}) es necesariamente un estado producto $\rho_A\otimes\rho_B$,
con $\rho_A$ y $\rho_B$ puros.

\section{Criterios Básicos de Separabilidad} 
En general, excepto en casos simples como el de dos qubits, no es f\'acil determinar si un estado no puro,
es separable o entrelazado. En realidad es un problema considerado en general `` {\it hard} '' \cite{Gr.01}.

El {\bf criterio de la traspuesta parcial}, introducido por Asher Peres en 1996
\cite{P.96}, proporciona un criterio de separabilidad simple, computable y necesario, pero en general no suficiente. Es decir,
\begin{equation}
\rho_{AB}\;{separable}\;\Rightarrow
\rho_{AB}^{t_A}\geq 0\,,
\end{equation}
%o sea,
%\begin{equation}
%\rho_{AB}^{t_A}\,<\,0\;\Rightarrow
% \rho_{AB}\;{entrelazado}\;\,\end{equation}
donde $t^A$ denota {\it trasposición parcial} \cite{NC.00}
($\langle ij|\rho_{AB}^{t_A}|kl\rangle=\langle kj|\rho_{AB}|il\rangle$).
%en una base producto). 
Es decir, si $\rho_{AB}^{t_A}$ tiene algún autovalor negativo entonces $\rho_{AB}$ es entrelazado.
Pero si todos sus autovalores son no-negativos puede ser aún entrelazado. S\'olo
en el caso de dos qubits o qubit/qutrit, el presente criterio es {\it
necesario y suficiente} \cite{P.96,Ho.96}.


Interpretémoslo de otra forma: 
Si tenemos $\PROYECT{ij}{kl}=\PROYECT ik\otimes\PROYECT jl$ entonces
\[
\PARENTESIS{\PROYECT{ij}{kl}}^{T_{B}}=\PROYECT{il}{kj}=\PROYECT ik\otimes\PROYECT lj
\]

Si $\rho_{AB}$ es separable entonces 
\[
\rho_{AB}^{T_{B}}=\sum_{\alpha}p_{\alpha}\PARENTESIS{\rho_{A}^{\alpha}\otimes\rho_{B}^{\alpha}}^{T_{B}}=\sum_{\alpha}p_{\alpha}\rho_{A}^{\alpha}\otimes\PARENTESIS{\rho_{B}^{\alpha}}^{T}
\]

La traspuesta de $\rho_{B}^{\alpha}$ no cambia los autovalores, entonces
la traza sigue siendo 1. $\PARENTESIS{\rho_{B}^{\alpha}}^{T}=\rho_{B}^{\alpha}$,
$\PARENTESIS{\PARENTESIS{\rho_{B}^{\alpha}}^{T}}^{T}=\PARENTESIS{\rho_{B}^{\alpha}}^{T}$,
$Tr\PARENTESIS{\rho_{B}^{\alpha}}{}^{T}=Tr\rho_{B}^{\alpha}$

Ej: 
\[
\KET{\psi_{AB}}=\frac{\KET{00}+\KET{11}}{\sqrt{2}}
\]

\[
\rho_{AB}=\PROYECT{\psi_{AB}}{\psi_{AB}}=\frac{1}{2}\PARENTESIS{\LLAVEABAJO{\overset{}{\overset{\KET 0\BRA 0\otimes\KET 0\BRA 0}{\PROYECT{00}{00}}+\overset{\KET 1\BRA 1\otimes\KET 1\BRA 1}{\PROYECT{11}{11}}}}{\text{esto es separable }}+\PROYECT{00}{11}+\PROYECT{11}{00}}
\]

\[
\rho_{AB}^{T_{B}}=\frac{1}{2}\PARENTESIS{\PROYECT{00}{00}+\PROYECT{11}{11}+\PROYECT{01}{10}+\PROYECT{10}{01}}
\]

\[
\rho_{AB}^{T_{B}}=\frac{1}{2}\left(\begin{array}{cccc}
1 & 0 & 0 & 0\\
0 & 0 & 1 & 0\\
0 & 1 & 0 & 0\\
0 & 0 & 0 & 1
\end{array}\right)
\]

vemos arriba como era $\rho_{AB}$ sin transponer y vemos como afecta.
Los autovalores son

\[
\lambda\PARENTESIS{\rho_{AB}^{T_{B}}}=\PARENTESIS{\frac{1}{2},\frac{1}{2},\frac{1}{2},-\frac{1}{2}}
\]
cuando aplicamos la traza parcial a un estado entrelazado obtenemos
un autovalor negativo, entonces aplicar la traza parcial me genera
un $\rho_{AB}^{T_{B}}$ no es un operador densidad.

Resumen: el criterio me dice que si $\rho_{AB}$ es separable entonces
$\rho_{AB}^{T_{B}}\protect\geq 0$ entonces $\lambda\protect\PARENTESIS{\rho_{AB}^{T_{B}}}\protect\geq0$ 
(es decir sigue siendo un operador densidad). El contrarecíproco sería
que  $\rho_{AB}^{T_{B}}\cancel{\protect\geq}\,0$  es decir 
$\exists\lambda_{i}\protect\PARENTESIS{\rho_{AB}^{T_{B}}}<0$,
entonces $\rho_{AB}$ es entrelazado.





%\section
El {\bf criterio entrópico estándar} se basa en que los estados separables al 
igual que los sistemas clásicos, son siempre más
desordenados globalmente que localmente \cite{Ho.96}:
\begin{equation}
 \rho\;{\rm separable}\;\Rightarrow S(A,B)\geq S(A)\,, \label{CS}\end{equation}
y análogamente, $S(A,B)\geq S(B)$. Corresponden pues a entropías condicionales
$S(A|B)$ y $S(B|A)$ {\it positivas}. 

Los estados entrelazados  pueden satisfacer, como vimos, $S(A,B)<S(A)$, pero a diferencia del caso puro, en el caso no puro esta condición no es necesaria: Existen también estados
entrelazados que son más desordenados globalmente que localmente ($S(A,B)>S(A)$,
$S(A,B)>S(B)$). 
%en los que %las correlaciones cuánticas no se manifiestan pues a nivel entr\'opico. 
Notemos también que en el caso no puro, $S(A)$ no es necesariamente igual a $S(B)$.

El presente criterio entrópico ($\rho_{AB}$ separable $\Rightarrow$
$S(A,B)\geq S(A)$) puede generalizarse en realidad a otras entropías (por ejemplo, del tipo
$S(\rho)={\rm Tr}f(\rho)$, con $f$ c\'oncava y $f(0)=f(1)=0$ \cite{CR.02}), dando lugar al {\bf criterio entrópico generalizado} \cite{RC.02}, que es más
fuerte que el criterio entrópico basado en la entropía de von Neumann
\cite{Ho.96} y equivalente al criterio general de desorden \cite{NK.01}. 

\section{Medidas de Entrelazamiento}

La medida de entrelazamiento en estados no puros es un tema que no está
cerrado y es aún más difícil. Usualmente se utiliza como medida el entrelazamiento de formación, 
definido por la denominada ``{\it Convex Roof Extension}'' de la definición para
estados puros \cite{V.07,BDSW.96}:
\begin{equation}
 E(A,B)\equiv E(\rho_{AB})=\mathop{\rm Min}_{\sum_i q_i|\Psi_i\rangle\langle \Psi_i|=\rho_{AB}}\,
 \sum_i q_i\,E(|\Psi_i\rangle\langle\Psi_i|)\label{EABG}\end{equation}
es decir, es el mínimo, entre todas las representaciones posibles  de
$\rho_{AB}$ como combinación convexa de estados puros $|\Psi_i\rangle$ (no necesariamente ortogonales), del promedio del entrelazamiento en los mismos, 
 definido de acuerdo a (\ref{EAB}). En general, la cantidad (\ref{EABG})
no es computable de forma exacta. 

\subsection{Concurrencia \label{Concu}}
La gran excepción es el caso de dos qubits (o sea, dos sistemas con espacio de Hilbert 
local de dimensión $2$, tal como un par de espines $1/2$), donde W.K. Wootters logró obtener una fórmula general computable en 1998 por
medio de la llamada concurrencia $C_{AB}$ \cite{W.98}: 
\begin{eqnarray}
E(A,B)&=&-\sum_{\nu=\pm}q_\nu\log  q_\nu,\label{EABC}\end{eqnarray}
donde 
\begin{eqnarray}
q_\nu&=&{\textstyle\frac{1\pm\sqrt{1-C^2(A,B)}}{2}}\,,\label{qEABC}\\
C(A,B)&=&{\rm Max}[2\lambda_M-{\rm Tr}R,0]\,.
\label{CAB}\end{eqnarray}
Aquí $\lambda_M$ es el autovalor m\'aximo de la matriz
$R=\sqrt{\rho_{AB}^{1/2}\tilde{\rho}_{AB} \rho_{AB}^{1/2}}$, con
$\tilde{\rho}_{AB}=\sigma_y\otimes
\sigma_y\rho_{AB}^{*}\sigma_y\otimes\sigma_y$ en la base estándar, 
compuesta por los autoestados producto de $\sigma_{z}\otimes\sigma_z$. Aquí
$\bm{\sigma}=(\sigma_x,\sigma_y,\sigma_z)$ denota las matrices de Pauli. 

Se verifica 
\begin{equation}
0\leq C(A,B)\leq 1\,,\;\;0\leq E(A,B)\leq 1
\end{equation}
 con $E(A,B)=C(A,B)=1$  para  un estado de Bell (que es, por lo tanto, un estado m\'aximamente entrelazado), y 
$E(A,B)=C(A,B)=0$ para un estado separable, siendo
$E(A,B)$ una función estrictamente creciente de $C(A,B)$. 
%Puede pues utilizarse tambi\'en $C(A,B)$ como medida de entrelazamiento.

Para el caso de un estado puro arbitrario de dos qubits, se ve que
(\ref{EABC}) se reduce a la entropía $S(A)=S(B)$ de cualquiera de los
qubits, dada por la expresión (\ref{EAB2}) con $n_s=2$.
En tal caso $C(A,B)=2\sqrt{\sigma_1\sigma_2}$.

\subsection{Negatividad \label{negatividad}}
La negatividad es un estimador de entrelazamiento computable para estados mixtos de cualquier dimensión \cite{VW.02,ZHSL.99,ZHSL.99b}, definida por 

\begin{equation}
N_{AB}=({\rm Tr}\,|\rho_{AB}^{\rm t_A}|-1)/2\label{N12}\,,\end{equation}
donde $\rho_{AB}^{\rm t_2}$ denota la traspuesta parcial de $\rho_{AB}$. La Ec.\ (\ref{N12})
es simplemente el valor absoluto de la suma de los autovalores negativos de $\rho_{AB}^{\rm t_A}$. 
Si $\rho_{AB}$ es un estado puro   ($\rho_{AB}=|\psi_0\rangle\langle\psi_0|$), la  Ec.\ (\ref{N12})
se reduce a una entropía de entrelazamiento generalizada,
\begin{equation}N_{AB}=[({\rm Tr}\,\sqrt{\rho_A})^2-1]/2
 =\sum_{i<j}\lambda_i^1\lambda^1_j\label{N12p}\end{equation}
donde $\rho_A={\rm Tr}_B\,|\psi_0\rangle\langle\psi_0|$ 
es el estado reducido de $A$  y $\lambda^1_i$ sus autovalores. En este caso el estado es entrelazado si y solo si $N_{AB}>0$. Consecuentemente, la Ec.\
(\ref{N12p}) se anula para $\rho_A$ puro ($|\psi_0\rangle$ separable), y
alcanza su máximo para $\rho_A$  máximamente mezclado (es decir, $|\psi_0\rangle$ máximamente entrelazado),
en cuyo caso $N_{AB}=(d-1)/2$, con $d={\rm Min}[d_A,d_B]$ (en particular, $N_{AB}=s$  para un par de espines $s$). 

En el caso mixto general, $N_{AB}>0$ implica entrelazamiento de $\rho_{AB}$, pero $N_{AB}=0$ no implica necesariamente separabilidad, salvo para sistemas qubit-qubit o qubit-qutrit \cite{Ho.96}, ya que existen ciertos estados mixtos entrelazados ({\it bound entangled states}) que igualmente cumplen $N_{AB}=0$. 
No obstante, dada su computabilidad, $N_{AB}$ es corrientemente utilizada como una medida o estimador de entrelazamiento de estados mixtos. 

Ejemplo: Sistema de 2 qubits

\[
\rho_{AB}=p\PROYECT{\psi_{AB}}{\psi_{AB}}+\PARENTESIS{1-p}\frac{I_{AB}}{4}
\]

$\frac{I_{AB}}{4}$ este es el estado máximamente mezclado.

\textbf{¿Cuál es el valor umbral de p para que exista entrelazamiento?}

\[
\rho_{AB}=\left(\begin{array}{cccc}
\frac{p}{2}+\frac{1-p}{4} & 0 & 0 & \frac{p}{2}\\
0 & \frac{1-p}{4} & 0 & 0\\
0 & 0 & \frac{1-p}{4} & 0\\
\frac{p}{2} & 0 & 0 & \frac{1-p}{4}
\end{array}\right)
\]

Si hacemos la transpuesta parcial, 
\[
\rho_{AB}^{T_{B}}=\left(\begin{array}{cccc}
\frac{p}{2}+\frac{1-p}{4} & 0 & 0 & 0\\
0 & \frac{1-p}{4} & \frac{p}{2} & 0\\
0 & \frac{p}{2} & \frac{1-p}{4} & 0\\
0 & 0 & 0 & \frac{1-p}{4}
\end{array}\right)
\]

\[
\lambda\PARENTESIS{\rho_{AB}^{T_{B}}}=\PARENTESIS{\frac{p}{2}+\frac{1-p}{4},\frac{p}{2}+\frac{1-p}{4},\frac{p}{2}\pm\frac{1-p}{4}}
\]

\[
\frac{1-p}{4}-\frac{p}{2}<0\rightarrow1-3p<0\rightarrow p>\frac{1}{3}
\]
\subsection{Testigo de entrelazamiento}

Un testigo de entrelazamiento es un operador autoadjunto $O_{AB}$
tal que $Tr\rho_{AB\,sep}O_{AB}\geq0$ $\forall\rho_{AB\:sep}$. Pero,
dado un cierto $\rho_{AB}^{E}$ entrelazado $Tr\rho_{AB}^{E}O_{AB}<0$
(esto vale solo para un cierto $\rho_{AB}^{E}$ no para todos).

\paragraph{Dado $\rho_{AB}$ entrelazado $\exists$ $O_{AB}$ tal que $Tr\rho_{AB}O_{AB}<0$,
pero la traza de cualquier operador separable $Tr\rho_{AB}^{S}O_{AB}\protect\geq0$
$\forall\rho_{AB}^{S}$. }

\textbf{Demostración}
\begin{center}
\includegraphics[scale=0.5]{\string"fig clase 1809\string".pdf}
\par
\end{center}

\textbf{1) }El conjunto de todos los $\rho$ de un sistema físico
es un conjunto convexo. Quiere decir que si $\rho_{1}$ es operador
densidad y $\rho_{2}$ es operador densidad entonces $p\rho_{1}+(1-p)\rho_{2}$
es un operador densidad $\forall p\in\CORCHETES{0,1}$. 

Entonces si supongamos que este es el conjunto de los operadores densidad
y el borde de los $\rho$ son los estados puros. Todo conjunto convexo
se pueden generar a partir de unos estados borde, estados límite.
Porque del punto de vista matemático un punto que está en el medio
entre dos estados de borde lo puedo escribir como combinación lineal
de dos estados de borde de la forma $p\rho_{1}+(1-p)\rho_{2}$.

Dentro del conjunto convexo de todos los operadores densidad de un
sistema dado. En el borde estados puros. Adentro está el conjunto
de los $\rho$ separables.

\textbf{2) }El conjunto de los operadores densidad separables para
un dado sistema físico es convexo. La robustes de un operador separable
que mezclas cosas y seguis estando ahí adentro no la tienen los operadores
producto, una mezcla de producto no es un operador producto. Por eso
el concepto de separabilidad es importante.

\[
\begin{array}{c}
\rho_{1}=\sum_{\alpha}p_{\alpha}^{1}\rho_{A}^{1\alpha}\otimes\rho_{B}^{1\alpha}\\
\rho_{2}=\sum_{\alpha}p_{\alpha}^{2}\rho_{A}^{2\alpha}\otimes\rho_{B}^{2\alpha}
\end{array}\Rightarrow q\rho_{1}+(1-q)\rho_{2}=\sum_{\alpha}qp_{\alpha}^{1}\rho_{A}^{1\alpha}\otimes\rho_{B}^{1\alpha}+(1-q)p_{\alpha}^{2}\rho_{A}^{2\alpha}\otimes\rho_{B}^{2\alpha}
\]

Esto sigue siendo un operador densidad separable, mientras $q\in\CORCHETES{0,1}$. 

\textit{Desde el punto de vista físico la propiedad de convexidad es importante
porque te esta diciendo que si yo genero algo en este conjunto y hago
mezclas de eso, es decir que por ej. si tiro un dado cuando en el
dado sale 1 genero un cierto estado separable, cuando me sale 2 otro,
y así, y el estado promedio que sale es un estado separable también.
Es decir mezclando esas cosas no me voy del conjunto. \\}

El conjunto de estados producto no es convexo, porque no son conmutantes.
Los estados separables tienen en general autoestados entrelazados,
ie el hecho que sea una combinación convexa de productos como estos
estados $\PARENTESIS{\rho_{AB}=\sum_{\alpha}p_{\alpha}\rho_{A}^{\alpha}\otimes\rho_{B}^{\alpha}}$
que estamos sumando, donde cada $\alpha$ es un índice de estado $\rho_{A}$,
$\rho_{B}$ cualquiera donde no tienen porque ser conmutantes, son
suma de productos no conmutantes. Al sumar todo esto al ser no conmutantes,
si uno analiza esto, no todos o una buena parte son entrelazados.
Pero aún así el estado separable es generable por operaciones locales
y conmunicación clásica LOCC. 

Notar que el conjunto de los operadores densidad $\rho$ se toca en
sólo punto con el de los $\rho_{s}$, es un punto donde es separable
y puro al mismo tiempo y es uno sólo porque \textbf{implica} que $\rho$
es un producto
\[
\rho=\KET{k_{A}}\KET{k_{A}}\otimes\BRA{k_{B}}\BRA{k_{B}}\rightarrow\rho=\KET{\psi_{AB}}\BRA{\psi_{AB}}
\]

Volvamos a la \textbf{demostración}, supongamos que tenemos un $\rho_{AB}$
entrelazado, y tenemos la recta roja que me divide el espacio en 2
semiplanos (sup que estamos en 2 d)
\begin{center}
\includegraphics[scale=0.5]{\string"fig clase 1809- 2\string".pdf}
\par
\end{center}

Esto pasa a $\GREEN{Tr\rho_{AB}O_{AB}\geq0}$ y a $\BLUE{Tr\rho_{AB}O_{AB}\geq0}$,
siendo los coeficientes $a,\:b,\:c$ sería el operador testigo $O_{AB}$
y las coordenadas $x,\:y$ son los elementos del operador $\rho_{AB}$
en una base de operadores y esta condición sobre $Tr\rho_{AB}O_{AB}$
es una proyección lineal de $\rho_{AB}$ sobre alguna base. El hecho
que $\rho^{s}$ sea convexo hace que exista una recta entre $\rho$
entrelazado y $\rho^{s}$, entonces se prende la <<lámparita>> cuando
estoy de uno de los lados.$\blacksquare$

Podemos ver que el entrelazamiento es muy débil porque ponele que
genero un estado que este del lado izquierdo que caiga fuera del conjunto
de los separables, eso no es un conjunto convexo entonces me va a
fallar. Es decir, en cuanto tenga probabilidad de generar un entrelazado
ahí se cae el entrelazamiento, y eso puede pasar si hay ruido. En
cambio con los separables no pasa eso.

\textbf{El conjunto de estados entrelazados no es convexo.}

\subsubsection{Ejemplo}\label{Ej.entrelazamiento}
Supongamos que tenemos los estados de Bell 

\[
\left\{ \begin{array}{c}
\KET{\psi_{\underset{01}{00}}}=\frac{\KET{00}\pm\KET{11}}{\sqrt{2}}\\
\KET{\psi_{\underset{11}{10}}}=\frac{\KET{01}\pm\KET{10}}{\sqrt{2}}
\end{array}\right\} \text{base de estados de 2 qubits de }\COMPLEJOS^{2}\otimes\COMPLEJOS^{2}
\]

\begin{center}
\begin{quantikz}  \lstick{$\ket{i}$} & \gate{H} & \ctrl{1} & \midstick[2,brackets=none]{$\ket{\psi_{ij}}$}\qw\\  \lstick{$\ket{j}$} & \qw & \octrl{-1} & \qw \end{quantikz} 
\par
\end{center}

Tenemos que el estado $\PROYECT{\psi_{ij}}{\psi_{ij}}$ es entrelazado
pero 
\[
\sum_{i,j=0}^{1}\PROYECT{\psi_{ij}}{\psi_{ij}}\frac{1}{4}=\frac{I_{AB}}{4}\text{ es el estado máximamente mezclado}
\]
Como la identidad es la identidad en todas las bases, entonces lo
anterior es igual a 

\[
=\frac{1}{4}\sum_{i,j=0}^{1}\KET i\KET j\BRA i\BRA j=\frac{1}{4}\PARENTESIS{\KET{00}\BRA{00}+\KET{01}\BRA{01}+\KET{10}\BRA{10}+\KET{11}\BRA{11}}
\]
genero un ruido al azar. Esto implica que las medidas de entrelazamiento
no son cóncavas sino son convexas ie el entrelazamiento de una mezcla
tiene que ser menor que la mezcla de entrelazamientos. El entrelazamiento
de un promedio siempre es menor o igual que el promedio de los entrelazamientos. 

\section{Fidelidad} 
La fidelidad \cite{NC.00} es una medida de la distancia entre dos estados cuánticos (puros o no puros).
Se define como 
\begin{equation}
F(\rho,\rho')={\rm Tr}\,\sqrt{\rho^{1/2}\rho'\rho^{1/2}}\,.
\end{equation}
Para estados puros $\rho=|\psi\rangle\langle\psi|$, 
$\rho'=|\psi'\rangle\langle\psi'|$, $F(\rho,\rho')$ se reduce al modulo del {\it overlap}:
\begin{equation}
F(\rho,\rho')=|\langle \psi|\psi'\rangle|\,.\end{equation}
En ambos casos, la fidelidad es un número entre $0$ y $1$,
\[0\leq F\leq 1\,,\]
con $F(\rho,\rho')=1$ si y solo si $\rho=\rho'$ y $F(\rho,\rho')=0$ si y solo si $\rho$ y $\rho'$ tienen soportes ortogonales. 

La fidelidad está relacionada con otra cantidad que mide cuan diferentes son dos estados, conocida como medida o m\'etrica de Wootters. Esta última puede evaluarse en función de la primera por la relación

\begin{equation}
B(\rho,\rho')=\arccos F(\rho,\rho') 
\label{metric}
\end{equation}

Esta medida define una distancia entre los operadores estadísticos, ya que es una cantidad semidefinida positiva y simétrica que satisface la desigualdad triangular 
($B(\rho,\rho')\leq B(\rho,\rho'')+B(\rho'',\rho')$). 

\subsection{Entrelazamiento de formación}

Miremos primero como calculo el valor medio de la energía

\[
\VALMEDIO H=Tr\rho H
\]
con $H$ un hamiltoniano, con $\rho=\sum_{i}p_{i}\PROYECT ii$, luego
$\VALMEDIO H=\EXPECT ii H=\sum_{i}\EXPECT ii H p_{i}$
implica un valor medio cuántico con distribución de probabilidad y
un valor medio clásico con $p_{i}$.

Nosotros sabemos que el entrelazamiento de un estado puro $E\PARENTESIS{\KET{\psi_{AB}}}=S\PARENTESIS{\rho_{A}}=S\PARENTESIS{\rho_{B}}$,
con $\rho_{A}=Tr_{B}\PROYECT{\psi_{AB}}{\psi_{AB}}$, entonces uno
podría definir (haciendo un razonamiento para comenzar, luego veremos que en realidad es incorrecto)
 $E\PARENTESIS{\rho_{AB}}=\sum_{i}p_{i}E\PARENTESIS{\KET{\psi_{AB}}}$,
esta definición no tiene sentido por el ejemplo que vimos recién de estados
de Bell \ref{Ej.entrelazamiento}, si aplico esta definición de entrelazamiento a la expresión
a una mezcla de estados de Bell me da 1 y si lo aplico a una mezcla
de estados producto me da 0, es decir me queda $0=1$. Tiene que ser
una propiedad del vector pero no debe depender de la representación
que elija al vector. 

Definimos al \textit{entrelazamiento de formación} como

\[
E\PARENTESIS{\rho_{AB}}=\underset{\LLAVES{p_{i}\KET{\psi_{AB}^{i}}/\sum_{i}p_{i}\PROYECT{\psi_{AB}^{i}}{\psi_{AB}^{i}}=\rho_{AB}}}{\text{Min}}\sum_{i}p_{i}E\PARENTESIS{\KET{\psi_{AB}^{i}}}
\]
 Tomo el mínimo sobre todas las representaciones. Esto es consistente
ya que cumple que $E\PARENTESIS{\rho_{AB}}\ge0$, $E\PARENTESIS{\rho_{AB}}=0\iff\rho_{AB}\text{ separable}$. 

Este no es un criterio necesario y suficiente porque está muy bien
pero no es computable (no es física, porque tenés que pensar sobre
todas las posibles mezclas que te da un estado, es un espacio infinito). 

\textbf{Caso de 2 qubits. Wootters}

Logró evaluar de forma analítica $E\PARENTESIS{\rho_{AB}}$ $\forall\rho_{AB}$
de 2 qubits.

\subsection{Fórmula de Wootters (2 qubits) }

\[
E\PARENTESIS{\rho_{AB}}=-\sum_{i=0,1}p_{i}\log p_{i}=S\PARENTESIS{\rho_{A}}=S\PARENTESIS{\rho_{B}}
\]

Donde $p_{\overset{0}{1}}=\frac{1\pm\sqrt{1-c^{2}}}{2}$, con $c$
la concurrencia. Siendo $c=2\lambda_{max}(R)-Tr(R)$ y $R=\sqrt{\rho_{AB}^{1/2}\tilde{\rho}_{AB}\rho_{AB}^{1/2}}$,
con $\tilde{\rho}_{AB}=\sigma_{y}\otimes\sigma_{y}\rho_{AB}^{*}\sigma_{y}\otimes\sigma_{y}$. 

Para \textbf{estados puros} $E\PARENTESIS{\rho_{AB}}=S\PARENTESIS{\rho_{A}}=S\PARENTESIS{\rho_{B}}$.
Para estados mezcla $S\PARENTESIS{\rho_{A}}\neq S\PARENTESIS{\rho_{B}}$
y no mide ni entrelazamiento ni correlación. 

Para estados puros la concurrencia la podemos calcular como $c\PARENTESIS{\rho_{AB}}=\sqrt{1-Tr\PARENTESIS{\rho_{AB}^{2}}}$,
es decir todas las medidas de correlación o entrelazamiento bien definidas
cuando son medidas para un estado puro deben reducirse a una medida
de entropía de estado local (no necesariamente la de Shannon). La
$Tr\PARENTESIS{\rho_{AB}^{2}}$ se llama pureza, ya que da 1 para
estados puros ya que $\rho_{AB}^{2}=\rho_{AB}$.




